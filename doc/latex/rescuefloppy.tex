%%
%%

\section*{Disaster Recovery Using a Bacula Rescue Floppy}
\label{_ChapterStart24}
\index[general]{Floppy!Disaster Recovery Using a Bacula Rescue }
\index[general]{Disaster Recovery Using a Bacula Rescue Floppy }
\addcontentsline{toc}{section}{Disaster Recovery Using a Bacula Rescue Floppy}

\subsection*{General}
\index[general]{General }
\addcontentsline{toc}{subsection}{General}

Please note that the Bacula Rescue Floppy is now deprecated and is 
and is replaced by the Bacula Rescue CDROM described in another
chapter of this manual.

When disaster strikes, you must have a plan, and you must have prepared in
advance otherwise the work of recovering your system and your files will be
considerably greater. For example, if you have not previously saved the
partitioning information for your hard disk, how can you properly rebuild it
if the disk must be replaced? 

Unfortunately, many of the steps one must take before and immediately after a
disaster are very operating system dependent. As a consequence, this chapter
will discuss in detail disaster recovery (also called Bare Metal Recovery) for
{\bf Linux} and {\bf Solaris}. For Solaris, the procedures are still quite
manual. For FreeBSD the same procedures may be used but they are not yet
developed. For Win32, no luck. Apparently an ``emergency boot'' disk allowing
access to the full system API without interference does not exist. 
\label{considerations}

\subsection*{Important Considerations}
\index[general]{Important Considerations }
\index[general]{Considerations!Important }
\addcontentsline{toc}{subsection}{Important Considerations}

Here are a few important considerations concerning disaster recovery that you
should take into account before a disaster strikes. 

\begin{itemize}
\item If the building which houses your computers burns down or is otherwise 
   destroyed, do you have off-site backup data? 
\item Disaster recovery is much easier if you have several machines. If  you
   have a single machine, how will you handle unforeseen events  if your only
   machine is down? 
\item Do you want to protect your whole system and use Bacula to  recover
   everything? or do you want to try to restore your system from  the original
   installation disks and apply any other updates and  only restore user files? 
\end{itemize}

\label{steps}

\subsection*{Steps to Take Before Disaster Strikes}
\index[general]{Steps to Take Before Disaster Strikes }
\index[general]{Strikes!Steps to Take Before Disaster }
\addcontentsline{toc}{subsection}{Steps to Take Before Disaster Strikes}

\begin{itemize}
\item Create a Bacula Rescue floppy for each of your Linux systems.  
\item Ensure that you always have a valid bootstrap file that is  saved to an
   alternate machine.  
\item If possible copy your catalog nightly to an alternate machine.  If you
   have a valid bootstrap file, this is not necessary, but  can be very useful if
   you do not want to reload everything. .  
\item Test using the Bacula Rescue floppy before you are forced to use  it in
   an emergency situation. 
   \end{itemize}

\label{floppy}

\subsection*{Bare Metal Floppy Recovery on Linux with a Bacula Floppy Rescue
Disk}
\index[general]{Disk!Bare Metal Floppy Recovery on Linux with a Bacula Floppy
Rescue }
\index[general]{Bare Metal Floppy Recovery on Linux with a Bacula Floppy
Rescue Disk }
\addcontentsline{toc}{subsection}{Bare Metal Floppy Recovery on Linux with a
Bacula Floppy Rescue Disk}

Since floppies are being used less and less, the Bacula Floppy rescue disk is
deprecated, which means that it is no longer really supported. For those of
you who have or need floppy rescue, we include the recovery instructions here
for your reference. 

The remainder of this section concerns recovering a {\bf Linux} computer using
a floppy, and parts of it relate to the Red Hat version of Linux. 

A so called ``Bare Metal'' recovery is one where you start with an empty hard
disk and you restore your machine. There are also cases where you may lose a
file or a directory and want it restored. Please see the previous chapter for
more details for those cases. 

Bare Metal Recovery assumes that you have the following four items for your
system: 

\begin{itemize}
\item An emergency boot disk allowing you to boot without a  hard disk  
\item A Bacula Rescue floppy containing your disk information and  a number of
   helpful scripts (described below) including a  statically linked version of
   the Bacula File daemon 
\item A full Bacula backup of your system possibly including  Incremental or
   Differential backups since the last Full  backup 
\item A second system running the Bacula Director, the Catalog,  and the
   Storage daemon. (this is not an absolute requirement,  but how to get around
   it is not yet documented here) 
\end{itemize}

\subsection*{Restrictions}
\index[general]{Restrictions }
\addcontentsline{toc}{subsection}{Restrictions}

In addition, to the above assumptions, the following conditions or
restrictions apply: 

\begin{itemize}
\item Linux only -- tested only on Red Hat, but should work on other Linuxes  
\item The scripts handle only SCSI and IDE disks  
\item All partitions will be recreated, but only {\bf ext2},  {\bf ext3}, and
   {\bf swap} partitions will be reformatted.  Any other partitions such as
   Windows FAT partitions will  not be formatted by the scripts, but you can do
it by hand  
\item You are using either {\bf lilo} or {\bf grub} as a boot  loader, and you
   know which one (not automatically detected)  
\item The partitioning and reformating scripts will *should* work with RAID 
   devices, but probably not with other ``complicated'' disk 
   partitioning/formating schemes. Please check them carefully. You  will
probably need to edit the scripts by hand to make them work. 
\end{itemize}

\subsection*{Directories}
\index[general]{Directories }
\addcontentsline{toc}{subsection}{Directories}

If you are building a self-contained Bacula Rescue CDROM, you will find the
necessary scripts in {\bf rescule/linux/cdrom} subdirectory of the Bacula
source code. 

If you wish to build the Bacula Rescue floppy disk, the scripts discussed
below can be found in the {\bf rescue/linux/floppy} subdirectory of the Bacula
source code. 

\subsection*{Preparation for a Bare Metal Recovery}
\index[general]{Recovery!Preparation for a Bare Metal }
\index[general]{Preparation for a Bare Metal Recovery }
\addcontentsline{toc}{subsection}{Preparation for a Bare Metal Recovery}

There are two things you should do immediately on all (Linux) systems for
which you wish to do a bare metal recovery: 

\begin{enumerate}
\item Create a system emergency boot disk or alternatively  a system
   installation boot floppy. This step can be skipped  if you have an
   Installation CDROM and your machine will boot  from CDROM (most modern
computers will).  
\item Create a Bacula Rescue floppy, which captures the current working  state
   of your computer and creates scripts to restore it. In  addition, it creates a
   statically linked version of the Bacula  File daemon (Client) program, which
is key to successfully restoring  from scratch. 
\end{enumerate}

\subsubsection*{Creating an Emergency Boot Disk}
\index[general]{Creating an Emergency Boot Disk }
\index[general]{Disk!Creating an Emergency Boot }
\addcontentsline{toc}{subsubsection}{Creating an Emergency Boot Disk}

Here you have several choices: 

\begin{itemize}
\item Create a tomsrtbt disk (any Linux)  
\item Create an emergency boot disk (any Linux I think) 
\item Create a Red Hat Installation disk (Red Hat specific)  
\item Others 
   \end{itemize}

\paragraph*{tomsrtbt:}

If you have created a Bacula Rescue CDROM, you can skip this section. 

If you *must* use a boot floppy, my preference is to create and use a {\bf
tomsrtbt} emergency boot disk because it gives you a very clean Linux
environment (with a 2.2 kernel) and the most utilities. See 
\elink{http://www.toms.net/rb/}{http://www.toms.net/rb/} for more details on
this. It is very easy to do and well worth the effort. However, I recommend
that you create both especially if you have non-standard hardware. You may
find that {\bf tomsrtbt} will not work with your network driver (he surely has
one, but you must explicitly put it on the disk), whereas the Linux rescue is
more likely to work. 

\paragraph*{Emergency Boot Disk:}

If you have created a Bacula Rescue CDROM, you can skip this section. 

To create a standard Linux emergency boot disk you must first know the name of
the kernel, which you can find with: 

\footnotesize
\begin{verbatim}
  ls -l /boot
\end{verbatim}
\normalsize

and looking on the {\bf vmlinux-...} line or alternative do an 

\footnotesize
\begin{verbatim}
 uname -a
\end{verbatim}
\normalsize

then become root and with a blank floppy in the drive, enter the following
command: 

\footnotesize
\begin{verbatim}
  mkbootdisk --device /dev/fd0 2.4.18-18
\end{verbatim}
\normalsize

where you replace ``2.4.18-18'' by your system name. 

This disk can then be booted and you will be in an environment with a number
of important tools available. Some disadvantages of this environment as
opposed to {\bf tomsrtbt} are that you must enter {\bf linux rescue} at the
boot prompt or the boot will fail without a hard disk; it requires a disk boot
image or a CDROM to be mounted, if the CDROM is released, you will loose a
large number of the tools. 

\paragraph*{Red Hat Installation Disk:}

If you have created a Bacula Rescue CDROM, you can skip this section. 

Specific to Red Hat Linux, is to create an Installation floppy, which can also
be used as an emergency boot disk. The advantage of this method is that it
works in conjunction with the installation CDROM and hence during the first
part of restoring the system, you have a much larger number of tools available
(on the CDROM). This can be extremely useful if you are not sure what really
happened and you need to examine your system in detail. 

To make a Red Hat Linux installation disk, do the following: 

\footnotesize
\begin{verbatim}
mount the Installation CDROM (/mnt/cdrom)
cd /mnt/cdrom/images
dd if=boot.img of=/dev/fd0 bs=1440k
\end{verbatim}
\normalsize

Now that you have either an emergency boot disk or an installation floppy, you
will be able to reboot your system in the absence of your hard disk or with a
damaged hard disk. This method has the same disadvantages compared to {\bf
tomsrtbt} disk as mentioned above for the Emergency Boot Disk. 

\subsubsection*{Creating a Bacula Rescue Disk}
\index[general]{Creating a Bacula Rescue Disk }
\index[general]{Disk!Creating a Bacula Rescue }
\addcontentsline{toc}{subsubsection}{Creating a Bacula Rescue Disk}

If you have created a Bacula Rescue CDROM, this step will be automatically
done for you. 

Simply having a boot disk is not sufficient to re-create things as they were.
To solve this problem, we will create a Bacula Rescue disk. Everything that
will be written to this disk will first be placed into the {\bf
\lt{}bacula-src\gt{}/rescue/linux} directory. 

The first step is while your system is up and running normally, you use a
Bacula script called {\bf getdiskinfo} to capture certain important
information about your hard disk configuration (partitioning, formatting,
mount points, ...). {\bf getdiskinfo} will also create a number of scripts
using the information found that can be used in an emergency to repartition
your disks, reformat them, and restore a statically linked version of the
Bacula file daemon so that your disk can be restored from within a minimal
boot environment. 

The first step is to run {\bf getdiskinfo} as follows: 

\footnotesize
\begin{verbatim}
   su
   cd <bacula-src>/rescue/linux
   ./getdiskinfo
\end{verbatim}
\normalsize

{\bf getdiskinfo} works for either IDE or SCSI drives and recognizes both ext2
and ext3 file systems. If you wish to restore other file systems, you will
need to modify the code. This script can be run multiple times, but really
only needs to be run once unless you change your hard disk configuration. 

Assuming you have a single hard disk on device /dev/hda, {\bf getdiskinfo}
will create the following files: 

\begin{description}

\item [partition.hda]
   \index[fd]{partition.hda }
   This file contains the shell commands  to repartition your hard disk drive
/dev/hda to the current  state. If you have additional drives (e.g. /dev/hdc),
you  will find one of these files for each drive.  DO NOT EXECUTE THIS SCRIPT
UNLESS YOU WANT YOUR HARD DISK  REPARTITIONED  

\item [format.hda]
   \index[fd]{format.hda }
   This file contains the shell commands that  will format each of the partitions
on your hard drive. It  knows about ext2, ext3, and swap partitions. All other
partitions,  you must manually format. It is recommended that any Microsoft 
partitions be partitioned with Microsoft's {\bf format} command  rather than
using Unix tools.  DO NOT EXECUTE THIS SCRIPT UNLESS YOU WANT YOUR HARD DISK 
REFORMATTED  

\item [mount\_drives]
   \index[fd]{mount\_drives }
   This script will mount all ext2 and ext3  drives that were previously mounted.
They will be mounted on  

/mnt/drive/. This is used just before running the statically  linked Bacula so
that it can access your drives for the restore.  

\item [restore\_bacula]
   \index[fd]{restore\_bacula }
   This script will restore the File daemon  from the Bacula Rescue disk.
Building the Bacula Rescue disk  will be described later. This will provide
your emergency boot  environment with a Bacula file daemon. Note, this is a
special  statically linked version of the file daemon (i.e. it does not  need
or use shared libraries).  

\item [start\_network]
   \index[fd]{start\_network }
   This script will start your network using  the simplest possible commands. You
will need to verify that the  IP address used in this script is correct. In
addition, if you  have several ethernet cards, you may need to make other
modifications  to this script.  

\item [sfdisk]
   \index[fd]{sfdisk }
   This is the program that will repartition your hard disk,  and it is normally
found in {\bf /sbin/sfdisk}. It is placed in  this directory so that it will
be included on the rescue disk as  it is not normally available with all
emergency boot environments.  

\item [sfdisk.gz]
   \index[dir]{sfdisk.gz }
   This is the version of sfdisk that works with  {\bf tomsrtbt}. The standard
sfdisk described above will not  run under tomsrtbt. 
\end{description}

The {\bf getdiskinfo} program (actually a shell script) will also create a
subdirectory named {\bf diskinfo}, which contains the following files: 

\footnotesize
\begin{verbatim}
df.bsi
disks.bsi
fstab.bsi
ifconfig.bsi
mount.bsi
mount.ext2.bsi
mount.ext3.bsi
mtab.bsi
route.bsi
sfdisk.disks.bsi
sfdisk.hda.bsi
sfdisk.make.hda.bsi
\end{verbatim}
\normalsize

Each of these files contains some important piece of information (sometimes
redundant) about your hard disk setup or your network. Normally, you will not
need this information, but it will be written to the Bacula Rescue disk just
in case. Since it is normally not used, we will leave it to you to examine
those files at your leisure. 

\paragraph*{Building a Static File Daemon:}

If you have created a Bacula Rescue CDROM, this step will be automatically
done for you. 

The second of the three steps in creating your Bacula Rescue disk is to build
a static version of the File daemon. Do so by either configuring Bacula as
follows or by allowing the {\bf make\_rescue\_disk} script described below
make it for you: 

\footnotesize
\begin{verbatim}
cd <bacula-src>
./configure <normal-options>
make
cd src/filed
make static-bacula-fd
strip static-bacula-fd
cp static-bacula-fd ../../rescue/linux/bacula-fd
cp bacula-fd.conf ../../rescue/linux
\end{verbatim}
\normalsize

Note, above, we built {\bf static-bacula-fd} and changed its name to {\bf
bacula-fd} when copying it to the rescue/linux directory. 

Finally, in \lt{}bacula-src\gt{}/rescue/linux, ensure that the
WorkingDirectory and PIDDirectory both point to reasonable locations on a
stripped down system. If you are using {\bf tomsrtbt} you will also want to
replace machine names with IP addresses since there is no resolver running.
With the Linux Rescue disk, network address mapping seems to work. Don't
forget that at the time this version of the Bacula File daemon runs, your file
system will not be restored. In my bacula-fd.conf, I use {\bf /var/working}. 

\paragraph*{Writing the Bacula Rescue Floppy:}

When you have everything you need (output of getdiskinfo, Bacula File daemon,
...), you create your rescue floppy by putting a blank tape into your floppy
disk drive and entering: 

\footnotesize
\begin{verbatim}
su
./make_rescue_disk
\end{verbatim}
\normalsize

This script will reformat the floppy and write everything in the current
directory and all files in the {\bf diskinfo} directory to the floppy. If you
supply the appropriate command line options, it will also build a static
version of the Bacula file daemon and copy it along with the configuration
file to the disk. Also using a command line option, you can make it write a
compressed tar file containing all the files whose names are in {\bf
backup.etc.list} to the floppy. The list as provided contains names of files
in {\bf /etc} that you might need in a disaster situation. It is not needed,
but in some cases such as a complex network setup, you may find it useful. 

\paragraph*{Options for make\_rescue\_disk:}

The following command line options are available for the make\_rescue\_disk
script: 

\footnotesize
\begin{verbatim}
Usage: make_rescue_disk
  -h, --help             print this message
  --make-static-bacula   make static File daemon and add to diskette
  --copy-static-bacula   copy static File daemon to diskette
  --copy-etc-files       copy files in etc list to diskette
\end{verbatim}
\normalsize

Briefly the options are: 

\begin{description}

\item [\verb{--{make-static-bacula]
   \index[fd]{\verb{--{make-static-bacula }
   If this option is specified, the  script will assume that you have already
configured and  built Bacula. It will then proceed to build a statically 
linked version and copy it along with bacula-fd.conf to the  current
directory, then write it to the rescue disk.  

\item [\verb{--{copy-static-bacula]
   \index[dir]{\verb{--{copy-static-bacula }
   If this option is given, the  script will assume that you already have a copy
of the  statically linked Bacula in the current directory named  {\bf
bacula-fd} as well as the configuration script. They  will then be written to
the rescue disk.  

\item [\verb{--{copy-etc-files]
   \index[fd]{\verb{--{copy-etc-files }
   If this option is specified, the  script will tar the files in {\bf
backup.etc.list} and write  them to the rescue disk. 
\end{description}

Please examine the contents of the rescue floppy to ensure that it has
everything you want and need. If not modify the scripts as necessary and
re-run it until it is correct. 

Now that you have both a system boot floppy and a Bacula Rescue floppy,
assuming you have a full backup of your system made by Bacula, you are ready
to handle nearly any kind of emergency restoration situation. 
\label{FloppyRestore}

\subsection*{Restoring Your Linux Client with a Floppy}
\index[general]{Restoring Your Linux Client with a Floppy }
\index[general]{Floppy!Restoring Your Linux Client with a }
\addcontentsline{toc}{subsection}{Restoring Your Linux Client with a Floppy}

Now, let's assume that your hard disk has just died and that you have replaced
it with an new identical drive. In addition, we assume that you have: 

\begin{enumerate}
\item A recent Bacula backup (Full plus Incrementals)  
\item An emergency boot floppy (preferably {\bf tomsrtbt})  
\item A Bacula Rescue Floppy Disk  
\item Your Bacula Director, Catalog, and Storage daemon running  on another
   machine on your local network. 
   \end{enumerate}

This is a relatively simple case, and later in this chapter, as time permits,
we will discuss how you might recover from a situation where the machine that
crashes is your main Bacula server (i.e. has the Director, the Catalog, and
the Storage daemon). 

You will take the following steps to get your system back up and running: 

\begin{enumerate}
\item Boot with your Emergency Floppy  
\item Mount your Bacula Rescue floppy  
\item Start the Network (local network)  
\item Re-partition your hard disk(s) as it was before  
\item Re-format your partitions  
\item Restore the Bacula File daemon (static version)  
\item Perform a Bacula restore of all your files  
\item Re-install your boot loader  
\item Reboot 
   \end{enumerate}

Now for the details ... 

\subsubsection*{Boot with your Emergency Floppy}
\index[general]{Floppy!Boot with your Emergency }
\index[general]{Boot with your Emergency Floppy }
\addcontentsline{toc}{subsubsection}{Boot with your Emergency Floppy}

First you will boot with your emergency floppy. If you use the Installation
floppy described above, when you get to the boot prompt: 

\footnotesize
\begin{verbatim}
boot:
\end{verbatim}
\normalsize

you enter {\bf linux rescue}. 

If you are booting from {\bf tomsrtbt} simply enter the default responses. 

When your machine finishes booting, you should be at the command prompt
possibly with your hard disk mounted on {\bf /mount/sysimage} (Linux emergency
only). To see what is actually mounted, use: 

\footnotesize
\begin{verbatim}
df
\end{verbatim}
\normalsize

\paragraph*{Mount your Bacula Rescue Floppy:}

Make sure that the mount point {\bf /mnt/floppy} exists. If not, enter: 

\footnotesize
\begin{verbatim}
mkdir -p /mnt/floppy
\end{verbatim}
\normalsize

the mount your {\bf Bacula Rescue} disk and cd to it with: 

\footnotesize
\begin{verbatim}
mount /dev/fd0 /mnt/floppy
cd /mnt/floppy
\end{verbatim}
\normalsize

To simplify running the scripts make sure the current directory is on your
path by: 

\footnotesize
\begin{verbatim}
PATH=$PATH:.
\end{verbatim}
\normalsize

\paragraph*{Start the Network:}

At this point, you should bring up your network. Normally, this is quite
simple and requires just a few commands. If you have booted from your Bacula
Rescue CDROM, please cd into the /bacula-hostname directory before continuing.
To simplify your task, we have created a script that should work in most cases
by typing: 

\footnotesize
\begin{verbatim}
./start_network
\end{verbatim}
\normalsize

You can test it by pinging another machine, or pinging your broken machine
machine from another machine. Do not proceed until your network is up. 

\paragraph*{Unmount Your Hard Disk (if mounted):}

When you are sure you want to repartition your disk, normally, if your disk
was damaged or if you are using {\bf tomsrtbt} your hard disk will not be
mounted. However, if it is you must first unmount it so that it is not in use.
Do so by entering {\bf df} and then enter the correct commands to unmount the
disks. For example: 

\footnotesize
\begin{verbatim}
umount /mnt/sysimage/boot
umount /mnt/sysimage/usr
umount /mnt/sysimage/proc
umount /mnt/sysimage/
\end{verbatim}
\normalsize

where you explicitly unmount ({\bf umount}) each sysimage partition and
finally, the last one being the root. Do another {\bf df} command to be sure
you successfully unmount all the sysimage partitions. 

This is necessary because {\bf sfdisk} will refuse to partition a disk that is
currently mounted. As mentioned, this should never be necessary with {\bf
tomsrtbt}. 

\paragraph*{Partition Your Hard Disk(s):}

If you are using {\bf tomsrtbt}, you will need to do the following steps to
get the correct sfdisk: 

\footnotesize
\begin{verbatim}
rm -f sfdisk
bzip2 -d sfdisk.bz2
\end{verbatim}
\normalsize

{\bf Do not do the above steps if you are using a standard Linux boot disk or
the Bacula Rescue CDROM.} 

Then proceed with partitioning your hard disk by: 

\footnotesize
\begin{verbatim}
./partition.hda
\end{verbatim}
\normalsize

If you have multiple disks, do the same for each of them. For SCSI disks, the
repartition script will be named: {\bf partition.sda}. If the script complains
about the disk being in use, simply go back and redo the {\bf df} command and
{\bf umount} commands until you no longer have your hard disk mounted. Note,
in many cases, if your hard disk was seriously damaged or a new one installed,
it will not automatically be mounted. If it is mounted, it is because the
emergency kernel found one or more possibly valid partitions. 

If for some reason this procedure does not work, you can use the information
in {\bf partition.hda} to re-partition your disks by hand using {\bf fdisk}. 

\paragraph*{Format Your Hard Disk(s):}

After partitioning your disk, you must format it appropriately. The formatting
script will put back swap partitions, normal Unix partitions (ext2) and
journaled partitions (ext3). Do so by entering for each disk: 

\footnotesize
\begin{verbatim}
./format.hda
\end{verbatim}
\normalsize

The format script will ask you if you want a block check done. We recommend to
answer yes, but realize that for very large disks this can take hours. 

\paragraph*{Mount the Newly Formatted Disks:}

Once the disks are partitioned and formatted, you can remount them with the
{\bf mount\_drives} script. All your drives must be mounted for Bacula to be
able to access them. Run the script as follows: 

\footnotesize
\begin{verbatim}
./mount_drives
df
\end{verbatim}
\normalsize

The {\bf df} will tell you if the drives are mounted. If not, re-run the
script again. It isn't always easy to figure out and create the mount points
and the mounts in the proper order, so repeating the {\bf ./mount\_drives}
command will not cause any harm and will most likely work the second time. If
not, correct it by hand before continuing. 

\paragraph*{Unmount the CDROM:}

Next, if you are using the Red Hat installation disk, unmount the CDROM drive
by doing: 

\footnotesize
\begin{verbatim}
umount /mnt/cdrom
\end{verbatim}
\normalsize

This is not necessary if you are running {\bf tomsrtbt}. In doing this, I find
it is always busy, and I haven't figured out how to unmount it (Linux boot
only). 

\paragraph*{Restore and Start the File Daemon:}

If you have booted with a Bacula Rescue CDROM, your statically linked Bacula
File daemon and the bacula-fd.conf file with be in the /bacula-hostname/bin
directory. Please skip the following paragraph and continue with editing the
Bacula configuration file. 

If you have not used a Bacula Rescue CDROM, now change (cd) to some directory
where you want to put the image of the Bacula File daemon. I use the tmp
directory my hard disk (mounted as {\bf /mnt/disk/tmp}) because it is easy.
Then install into the current directory Bacula by running the {\bf
restore\_bacula} script from the floppy drive. For example: 

\footnotesize
\begin{verbatim}
cd /mnt/disk
mkdir -p /mnt/disk/tmp
mkdir -p /mnt/disk/tmp/working
/mnt/floppy/restore_bacula
ls -l
\end{verbatim}
\normalsize

Make sure {\bf bacula-fd} and {\bf bacula-fd.conf} are both there. 

Edit the Bacula configuration file, create the working/pid/subsys directory if
you haven't already done so above, and start Bacula by entering: 

\footnotesize
\begin{verbatim}
chroot /mnt/disk /tmp/bacula-fd -c /tmp/bacula-fd.conf
\end{verbatim}
\normalsize

The above command starts the Bacula File daemon with your the proper root disk
location (i.e. {\bf /mnt/disk/tmp}. If Bacula does not start correct the
problem and start it. You can check if it is running by entering: 

\footnotesize
\begin{verbatim}
ps fax
\end{verbatim}
\normalsize

You can kill Bacula by entering: 

\footnotesize
\begin{verbatim}
kill -TERM <pid>
\end{verbatim}
\normalsize

where {\bf pid} is the first number printed in front of the first occurrence
of {\bf bacula-fd} in the {\bf ps fax} command. 

Now, you should be able to use another computer with Bacula installed to check
the status by entering: 

\footnotesize
\begin{verbatim}
status client=xxxx
\end{verbatim}
\normalsize

into the Console program, where xxxx is the name of the client you are
restoring. 

One common problem is that your {\bf bacula-dir.conf} may contain machine
addresses that are not properly resolved on the stripped down system to be
restored because it is not running DNS. This is particularly true for the
address in the Storage resource of the Director, which may be very well
resolved on the Director's machine, but not on the machine being restored and
running the File daemon. In that case, be prepared to edit {\bf
bacula-dir.conf} to replace the name of the Storage daemon's domain name with
its IP address. 

\paragraph*{Restoring using the RedHat Installation Disk:}

Suppose your system was damaged for one reason or another, so that the hard
disk and the partitioning and much of the filesystems are intact, but you want
to do a full restore. If you have booted into your system with the RedHat
Installation Disk by specifying {\bf linux rescue} at the {\bf boot:} prompt,
you will find yourself in a shell command with your disks already mounted (if
it was possible) in {\bf /mnt/sysimage}. In this case, you can do much like
you did above to restore your system: 

\footnotesize
\begin{verbatim}
cd /mnt/sysimage/tmp
mkdir -p /mnt/sysimage/tmp/working
/mnt/floppy/restore_bacula
ls -l
\end{verbatim}
\normalsize

Make sure that {\bf bacula-fd} and {\bf bacula-fd.conf} are both in the
current directory and that the directory names in the {\bf bacula-fd.conf}
correctly point to the appropriate directories. Then start Bacula with: 

\footnotesize
\begin{verbatim}
chroot /mnt/sysimage /tmp/bacula-fd -c /tmp/bacula-fd.conf
\end{verbatim}
\normalsize

\paragraph*{Restore Your Files:}

On the computer that is running the Director, you now run a {\bf restore}
command and select the files to be restored (normally everything), but before
starting the restore, there is one final change you must make using the {\bf
mod} option. You must change the {\bf Where} directory to be the root by using
the {\bf mod} option just before running the job and selecting {\bf Where}.
Set it to: 

\footnotesize
\begin{verbatim}
/
\end{verbatim}
\normalsize

then run the restore. 

You might be tempted to avoid using {\bf chroot} and running Bacula directly
and then using a {\bf Where} to specify a destination of {\bf /mnt/disk}. This
is possible, however, the current version of Bacula always restores files to
the new location, and thus any soft links that have been specified with
absolute paths will end up with {\bf /mnt/disk} prefixed to them. In general
this is not fatal to getting your system running, but be aware that you will
have to fix these links if you do not use {\bf chroot}. 

\paragraph*{Final Step:}

At this point, the restore should have finished with no errors, and all your
files will be restored. One last task remains and that is to write a new boot
sector so that your machine will boot. For {\bf lilo}, you enter the following
command: 

\footnotesize
\begin{verbatim}
run_lilo
\end{verbatim}
\normalsize

If you are using grub instead of lilo, you must enter the following: 

\footnotesize
\begin{verbatim}
run_grub
\end{verbatim}
\normalsize

Note, I've had quite a number of problems with {\bf grub} because it is rather
complicated and not designed to install easily under a simplified system. So,
if you experience errors or end up unexpectedly in a {\bf chroot} shell,
simply exit back to the normal shell and type in the appropriate commands from
the {\bf run\_grub} script by hand until you get it to install. 

\paragraph*{Reboot:}

Reboot your machine by entering {\bf exit} until you get to the main prompt
then enter {\bf ctl-d}. 

If everything went well, you should now be back up and running. If not,
re-insert the emergency boot floppy, boot, and figure out what is wrong. 

At this point, you will probably want to remove the temporary copy of Bacula
that you installed. Do so with: 

\footnotesize
\begin{verbatim}
rm -f /bacula-fd /bacula-fd.conf
rm -rf /working
\end{verbatim}
\normalsize

\label{FloppyProblems}

\subsection*{Linux Problems or Bugs}
\index[general]{Bugs!Linux Problems or }
\index[general]{Linux Problems or Bugs }
\addcontentsline{toc}{subsection}{Linux Problems or Bugs}

Since every flavor and every release of Linux is different, there are likely
to be some small difficulties with the scripts, so please be prepared to edit
them in a minimal environment. A rudimentary knowledge of {\bf vi} is very
useful. Also, these scripts do not do everything. You will need to reformat
Windows partitions by hand, for example. 

Getting the boot loader back can be a problem if you are using {\bf grub}
because it is so complicated. If all else fails, reboot your system from your
floppy but using the restored disk image, then proceed to a reinstallation of
grub (looking at the run-grub script can help). By contrast, lilo is a piece
of cake. 

\subsubsection*{Bugs}
\index[general]{Bugs }
\addcontentsline{toc}{subsubsection}{Bugs}

When performing the bare metal recovery using the Red Hat emergency boot disk
(actually the installation boot disk), I was never able to release the cdrom,
and when the system came up {\bf /mnt/cdrom} was soft linked to {\bf
/mnt/disk/dev/hdd}, which is not correct. I fixed this in each case by
deleting and simply remaking it with {\bf mkdir -p /mnt/cdrom}. 

\subsection*{tomsrtbt}
\index[general]{Tomsrtbt }
\addcontentsline{toc}{subsection}{tomsrtbt}

This is a single floppy (1.722Meg) that really has A LOT of software. For
example, by default (version 2.0.103) you get: 

AHA152X AHA1542 AIC7XXX BUSLOGIC DAC960 DEC\_ELCP(TULIP) EATA
EEXPRESS/PRO/PRO100 EL2 EL3 EXT2 EXT3 FAT FD IDE-CD/DISK/TAPE IMM INITRD
ISO9660 JOLIET LOOP MATH\_EMULATION MINIX MSDOS NCR53C8XX NE2000 NFS NTFS
PARPORT PCINE2K PCNET32 PLIP PPA RTL8139 SD SERIAL/\_CONSOLE SLIP SMC\_ULTRA
SR ST VFAT VID\_SELECT VORTEX WD80x3 .exrc 3c589\_cs agetty ash badblocks
basename boot.b buildit.s busybox bz2bzImage bzip2 cardmgr cardmgr.pid cat
chain.b chattr chgrp chmod chown chroot clear clone.s cmp common config cp
cpio cs cut date dd dd-lfs debugfs ddate df dhcpcd\verb{--{ dirname dmesg domainname
ds du dumpe2fs e2fsck echo egrep elvis ex false fdflush fdformat fdisk
filesize find findsuper fmt fstab grep group gunzip gzip halt head hexdump
hexedit host.conf hostname hosts httpd i82365 ifconfig ile init inittab insmod
install.s issue kernel key.lst kill killall killall5 ld ld-linux length less
libc libcom\_err libe2p libext2fs libtermcap libuuid lilo lilo.conf ln
loadkmap login ls lsattr lsmod lua luasocket man map md5sum miterm mkdir
mkdosfs mke2fs mkfifo mkfs.minix mknod mkswap more more.help mount mt mtab mv
nc necho network networks nmclan\_cs nslookup passwd pax pcmcia\_core
pcnet\_cs pidof ping poweroff printf profile protocols ps pwd rc.0 rc.S
rc.custom rc.custom.gz rc.pcmcia reboot rescuept reset resolv.conf rm rmdir
rmmod route rsh rshd script sed serial serial\_cs services setserial
settings.s sh shared slattach sleep sln sort split stab strings swapoff swapon
sync tail tar tcic tee telnet telnetd termcap test tomshexd tomsrtbt.FAQ touch
traceroute true tune2fs umount undeb\verb{--{ unpack.s unrpm\verb{--{ update utmp vi vi.help
view watch wc wget which xargs xirc2ps\_cs yecho yes zcat 

In addition, at 
\elink{Tom's Web Site}{http://www.toms.net/rb}, you can find a lot of
additional kernel drivers and other software (such as {\bf sdisk}, which is
used by Bacula. 

Building his floppy is a piece of cake. Simply download his .tar.gz file then:


\footnotesize
\begin{verbatim}
- detar the .tar.gz archive
- become root
- cd to the tomsrtbt-<version> directory
- load a blank floppy with no bad sectors
- ./install.s
\end{verbatim}
\normalsize
