%%
%%

\section*{Using Pools to Manage Volumes}
\label{_ChapterStart11}
\index[general]{Volumes!Using Pools to Manage }
\index[general]{Using Pools to Manage Volumes }
\addcontentsline{toc}{section}{Using Pools to Manage Volumes}

If you manage 5 or 10 machines and have a nice tape backup, you don't need
Pools, and you may wonder what they are good for. In this chapter, you will
see that Pools can help you optimize disk storage space. The same techniques
can be applied to a shop that has multiple tape drives, or that wants to mount
various different Volumes to meet their needs. 

The rest of this chapter will give an example involving backup to disk
Volumes, but most of the information applies equally well for tape Volumes. 
\label{TheProblem}

\subsection*{The Problem}
\index[general]{Problem }
\addcontentsline{toc}{subsection}{Problem}

A site that I administer (a charitable organization) had a tape DDS-3 tape
drive that was failing. The exact reason for the failure is still unknown.
Worse yet, their full backup size is about 15GB whereas the capacity of their
broken DDS-3 was at best 8GB (rated 6/12). A new DDS-4 tape drive and the
necessary cassettes was more expensive than their budget could handle. 
\label{TheSolution}

\subsection*{The Solution}
\index[general]{Solution }
\addcontentsline{toc}{subsection}{Solution}

They want to maintain 6 months of backup data, and be able to access the old
files on a daily basis for a week, a weekly basis for a month, then monthly
for 6 months. In addition, and offsite capability was not needed (well perhaps
it really is, but it was never used). Their daily changes amount to about
300MB on the average, or about 2GB per week. 

As a consequence, the total volume of data they need to keep to meet their
needs is about 100GB (15GB x 6 + 2GB x 5 + 0.3 x 7) = 102.1GB. 

The chosen solution was to buy a 120GB hard disk for next to nothing -- far
less than 1/10th the price of a tape drive and the cassettes to handle the
same amount of data, and to have Bacula write to disk files. 

The rest of this chapter will explain how to setup Bacula so that it would
automatically manage a set of disk files with the minimum intervention on my
part. The system has been running since 22 January 2004 until today (08 April
2004) with no intervention. Since we have not yet crossed the six month
boundary, we still lack some data to be sure the system performs as desired. 
\label{OverallDesign}

\subsection*{Overall Design}
\index[general]{Overall Design }
\index[general]{Design!Overall }
\addcontentsline{toc}{subsection}{Overall Design}

Getting Bacula to write to disk rather than tape in the simplest case is
rather easy, and is documented in the previous chapter. In addition, all the
directives discussed here are explained in that chapter. We'll leave it to you
to look at the details there. If you haven't read it and are not familiar with
Pools, you probably should at least read it once quickly for the ideas before
continuing here. 

One needs to consider about what happens if we have only a single large Bacula
Volume defined on our hard disk. Everything works fine until the Volume fills,
then Bacula will ask you to mount a new Volume. This same problem applies to
the use of tape Volumes if your tape fills. Being a hard disk and the only one
you have, this will be a bit of a problem. It should be obvious that it is
better to use a number of smaller Volumes and arrange for Bacula to
automatically recycle them so that the disk storage space can be reused. The
other problem with a single Volume, is that at the current time (1.34.0)
Bacula does not seek within a disk Volume, so restoring a single file can take
more time than one would expect. 

As mentioned, the solution is to have multiple Volumes, or files on the disk.
To do so, we need to limit the use and thus the size of a single Volume, by
time, by number of jobs, or by size. Any of these would work, but we chose to
limit the use of a single Volume by putting a single job in each Volume with
the exception of Volumes containing Incremental backup where there will be 6
jobs (a week's worth of data) per volume. The details of this will be
discussed shortly. 

The next problem to resolve is recycling of Volumes. As you noted from above,
the requirements are to be able to restore monthly for 6 months, weekly for a
month, and daily for a week. So to simplify things, why not do a Full save
once a month, a Differential save once a week, and Incremental saves daily.
Now since each of these different kinds of saves needs to remain valid for
differing periods, the simplest way to do this (and possibly the only) is to
have a separate Pool for each backup type. 

The decision was to use three Pools: one for Full saves, one for Differential
saves, and one for Incremental saves, and each would have a different number
of volumes and a different Retention period to accomplish the requirements. 
\label{FullPool}

\subsubsection*{Full Pool}
\index[general]{Pool!Full }
\index[general]{Full Pool }
\addcontentsline{toc}{subsubsection}{Full Pool}

Putting a single Full backup on each Volume, will require six Full save
Volumes, and a retention period of six months. The Pool needed to do that is: 

\footnotesize
\begin{verbatim}
Pool {
  Name = Full-Pool
  Pool Type = Backup
  Recycle = yes
  AutoPrune = yes
  Volume Retention = 6 months
  Accept Any Volume = yes
  Maximum Volume Jobs = 1
  Label Format = Full-
  Maximum Volumes = 6
}
\end{verbatim}
\normalsize

Since these are disk Volumes, no space is lost by having separate Volumes for
each backup (done once a month in this case). The items to note are the
retention period of six months (i.e. they are recycled after 6 months), that
there is one job per volume (Maximum Volume Jobs = 1), the volumes will be
labeled Full-0001, ... Full-0006 automatically. One could have labeled these
manual from the start, but why not use the features of Bacula. 
\label{DiffPool}

\subsubsection*{Differential Pool}
\index[general]{Pool!Differential }
\index[general]{Differential Pool }
\addcontentsline{toc}{subsubsection}{Differential Pool}

For the Differential backup Pool, we choose a retention period of a bit longer
than a month and ensure that there is at least one Volume for each of the
maximum of five weeks in a month. So the following works: 

\footnotesize
\begin{verbatim}
Pool {
  Name = Diff-Pool
  Pool Type = Backup
  Recycle = yes
  AutoPrune = yes
  Volume Retention = 40 days
  Accept Any Volume = yes
  Maximum Volume Jobs = 1
  Label Format = Diff-
  Maximum Volumes = 6
}
\end{verbatim}
\normalsize

As you can see, the Differential Pool can grow to a maximum of six volumes,
and the Volumes are retained 40 days and there after can be recycled. Finally
there is one job per volume. This, of course, could be tightened up a lot, but
the expense here is a few GB which is not too serious. 
\label{IncPool}

\subsubsection*{Incremental Pool}
\index[general]{Incremental Pool }
\index[general]{Pool!Incremental }
\addcontentsline{toc}{subsubsection}{Incremental Pool}

Finally, here is the resource for the Incremental Pool: 

\footnotesize
\begin{verbatim}
Pool {
  Name = Inc-Pool
  Pool Type = Backup
  Recycle = yes
  AutoPrune = yes
  Volume Retention = 20 days
  Accept Any Volume = yes
  Maximum Volume Jobs = 6
  Label Format = Inc-
  Maximum Volumes = 5
}
\end{verbatim}
\normalsize

We keep the data for 20 days rather than just a week as the needs require. To
reduce the proliferation of volume names, we keep a week's worth of data (6
incremental backups) in each Volume. In practice, the retention period should
be set to just a bit more than a week and keep only two or three volumes
instead of five. Again, the lost is very little and as the system reaches the
full steady state, we can adjust these values so that the total disk usage
doesn't exceed the disk capacity. 
\label{Example}

\subsection*{The Actual Conf Files}
\index[general]{Files!Actual Conf }
\index[general]{Actual Conf Files }
\addcontentsline{toc}{subsection}{Actual Conf Files}

The following example shows you the actual files used, with only a few minor
modifications to simplify things. 

The Director's configuration file is as follows: 

\footnotesize
\begin{verbatim}
Director {          # define myself
  Name = bacula-dir
  DIRport = 9101
  QueryFile = "/home/bacula/bin/query.sql"
  WorkingDirectory = "/home/bacula/working"
  PidDirectory = "/home/bacula/working"
  Maximum Concurrent Jobs = 1
  Password = " "
  Messages = Standard
}
#   By default, this job will back up to disk in /tmp
Job {
  Name = client
  Type = Backup
  Client = client-fd
  FileSet = "Full Set"
  Schedule = "WeeklyCycle"
  Storage = File
  Messages = Standard
  Pool = Default
  Full Backup Pool = Full-Pool
  Incremental Backup Pool = Inc-Pool
  Differential Backup Pool = Diff-Pool
  Write Bootstrap = "/home/bacula/working/client.bsr"
  Priority = 10
}
# List of files to be backed up
FileSet {
  Name = "Full Set"
  Include = signature=SHA1 compression=GZIP9 {
    /
    /usr
    /home
  }
  Exclude = {
     /proc /tmp /.journal /.fsck
  }
}
Schedule {
  Name = "WeeklyCycle"
  Run = Full 1st sun at 1:05
  Run = Differential 2nd-5th sun at 1:05
  Run = Incremental mon-sat at 1:05
}
Client {
  Name = client-fd
  Address = client
  FDPort = 9102
  Catalog = MyCatalog
  Password = " "
  AutoPrune = yes      # Prune expired Jobs/Files
  Job Retention = 6 months
  File Retention = 60 days
}
Storage {
  Name = File
  Address = localhost
  SDPort = 9103
  Password = " "
  Device = FileStorage
  Media Type = File
}
Catalog {
  Name = MyCatalog
  dbname = bacula; user = bacula; password = ""
}
Pool {
  Name = Full-Pool
  Pool Type = Backup
  Recycle = yes           # automatically recycle Volumes
  AutoPrune = yes         # Prune expired volumes
  Volume Retention = 6 months
  Accept Any Volume = yes # write on any volume in the pool
  Maximum Volume Jobs = 1
  Label Format = Full-
  Maximum Volumes = 6
}
Pool {
  Name = Inc-Pool
  Pool Type = Backup
  Recycle = yes           # automatically recycle Volumes
  AutoPrune = yes         # Prune expired volumes
  Volume Retention = 20 days
  Accept Any Volume = yes
  Maximum Volume Jobs = 6
  Label Format = Inc-
  Maximum Volumes = 5
}
Pool {
  Name = Diff-Pool
  Pool Type = Backup
  Recycle = yes
  AutoPrune = yes
  Volume Retention = 40 days
  Accept Any Volume = yes
  Maximum Volume Jobs = 1
  Label Format = Diff-
  Maximum Volumes = 6
}
Messages {
  Name = Standard
  mailcommand = "bsmtp -h mail.domain.com -f \"\(Bacula\) %r\"
      -s \"Bacula: %t %e of %c %l\" %r"
  operatorcommand = "bsmtp -h mail.domain.com -f \"\(Bacula\) %r\"
      -s \"Bacula: Intervention needed for %j\" %r"
  mail = root@domain.com = all, !skipped
  operator = root@domain.com = mount
  console = all, !skipped, !saved
  append = "/home/bacula/bin/log" = all, !skipped
}
\end{verbatim}
\normalsize

and the Storage daemon's configuration file is: 

\footnotesize
\begin{verbatim}
Storage {               # definition of myself
  Name = bacula-sd
  SDPort = 9103       # Director's port
  WorkingDirectory = "/home/bacula/working"
  Pid Directory = "/home/bacula/working"
}
Director {
  Name = bacula-dir
  Password = " "
}
Device {
  Name = FileStorage
  Media Type = File
  Archive Device = /files/bacula
  LabelMedia = yes;    # lets Bacula label unlabeled media
  Random Access = Yes;
  AutomaticMount = yes;   # when device opened, read it
  RemovableMedia = no;
  AlwaysOpen = no;
}
Messages {
  Name = Standard
  director = bacula-dir = all
}
\end{verbatim}
\normalsize

