%%
%%

\section*{Bacula\raisebox{.6ex}{{\footnotesize
\textsuperscript{\textregistered}}} - RPM Packaging FAQ}
\label{_ChapterStart34}
\index[general]{FAQ!Bacula\textsuperscript{\textregistered} - RPM Packaging }
\index[general]{Bacula\textsuperscript{\textregistered} - RPM Packaging FAQ }
\addcontentsline{toc}{section}{Bacula\textsuperscript{\textregistered} - RPM
Packaging FAQ}

\begin{enumerate}
\item 
   \ilink{How do I build Bacula for platform xxx?}{faq1}  
\item 
   \ilink{How do I control which database support gets built?}{faq2} 

\item 
   \ilink{What other defines are used?}{faq3}  
\item 
   \ilink{I'm getting errors about not having permission when I try to build the
   packages. Do I need to be root?}{faq4}  
\item 
   \ilink{I'm building my own rpms but on all platforms and compiles I get an
   unresolved dependancy for something called
/usr/afsws/bin/pagsh.}{faq5} 
\end{enumerate}

\subsection*{Answers}
\index[general]{Answers }
\addcontentsline{toc}{subsection}{Answers}

\begin{enumerate}
\item 
   \label{faq1}
   {\bf How do I build Bacula for platform xxx?}
The bacula spec file contains defines to build for several platforms:  RedHat
7.x (rh7), RedHat 8.0 (rh8), RedHat 9 (rh9), Fedora Core (fc1),  Whitebox
Enterprise Linux (RHEL) 3.0 (wb3), Mandrake 10.x (mdk) and SuSE 9.x (su9). 
The package build is controlled by a mandatory define set at the beginning of 
the file. These defines basically just control the dependancy information that
 gets coded into the finished rpm package. 
The platform define may be edited in the spec file directly (by default all 
defines are set to 0 or ``not set''). For example, to build the RedHat 7.x 
package find the line in the spec file which reads  

\footnotesize
\begin{verbatim}
        %define rh7 0
        
\end{verbatim}
\normalsize

and edit it to read  

\footnotesize
\begin{verbatim}
        %define rh7 1
        
\end{verbatim}
\normalsize

Alternately you may pass the define on the command line when calling rpmbuild:
 

\footnotesize
\begin{verbatim}
        rpmbuild -ba --define "build_rh7 1" bacula.spec
        rpmbuild --rebuild --define build_rh7 1" bacula-x.x.x-x.src.rpm
        
\end{verbatim}
\normalsize

\item 
   \label{faq2}
   {\bf How do I control which database support gets built?}
Another mandatory build define controls which database support is compiled,
one of  build\_sqlite, build\_mysql or build\_postgresql. To get the MySQL
package and support either  set the  

\footnotesize
\begin{verbatim}
        %define mysql 0
        
\end{verbatim}
\normalsize

to  

\footnotesize
\begin{verbatim}
        %define mysql 1
        
\end{verbatim}
\normalsize

in the spec file directly or pass it to rpmbuild on the command line:  

\footnotesize
\begin{verbatim}
        rpmbuild -ba --define "build_rh7 1" --define "build_mysql 1" bacula.spec
        
\end{verbatim}
\normalsize

\item 
   \label{faq3}
   {\bf What other defines are used?}
Two other building defines of note are the depkgs\_version and tomsrtbt
identifiers. These  two defines are set with each release and must match the
version of those sources that are  being used to build the packages. You would
not ordinarily need to edit these.  
\item 
   \label{faq4}
   {\bf I'm getting errors about not having permission when I try  to build the
packages. Do I need to be root?}
No, you do not need to be root and, in fact, it is better practice to build
rpm packages  as a non-root user. Bacula packages are designed to be built by
a regular user but you must  make a few changes on your system to do this. If
you are building on your own system then  the simplest method is to add write
permissions for all to the build directory  (/usr/src/redhat/). To accomplish
this execute the following command as root:  

\footnotesize
\begin{verbatim}
        chmod -R 777 /usr/src/redhat
        
\end{verbatim}
\normalsize

If you are working on a shared system where you can not use the method above
then you need to  recreate the /usr/src/redhat directory tree with all of it's
subdirectories inside your home  directory. Then create a file named  {\tt
.rpmmacros} in your home directory (or edit  the file if it already exists)
and add the following line:  

\footnotesize
\begin{verbatim}
        %_topdir /home/myuser/redhat
        
\end{verbatim}
\normalsize

\item 
   \label{faq5}
   {\bf I'm building my own rpms but on all platforms and compiles  I get an
unresolved dependancy for something called /usr/afsws/bin/pagsh.}
This is a shell from the OpenAFS (Andrew File System). If you are seeing this
then you  chose to include the docs/examples directory in your package. One of
the example scripts  in this directory is a pagsh script. Rpmbuild, when
scanning for dependancies, looks at  the shebang line of all packaged scripts
in addition to checking shared libraries. To avoid  this do not package the
examples directory.  
\end{enumerate}
