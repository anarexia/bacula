%%
%%

\section*{Installing Bacula}
\label{_ChapterStart17}
\index[general]{Bacula!Installing }
\index[general]{Installing Bacula }
\addcontentsline{toc}{section}{Installing Bacula}

\subsection*{General}
\index[general]{General }
\addcontentsline{toc}{subsection}{General}

In general, you will need the Bacula source release, and if you want to run a
Windows client, you will need the Bacula Windows binary release. However,
Bacula needs certain third party packages (such as {\bf SQLite}, {\bf MySQL}
to build properly depending on the options you specify. To simplify your task,
we have combined a number of these packages into two {\bf depkgs} releases
(Dependency Packages). This can vastly simplify your life by providing you
with all the necessary packages rather than requiring you to find them on the
Web, load them, and install them. 
\label{upgrading1}

\subsection*{Upgrading Bacula}
\index[general]{Bacula!Upgrading }
\index[general]{Upgrading Bacula }
\addcontentsline{toc}{subsection}{Upgrading Bacula}

If you are upgrading from one Bacula version to another, you should first
carefully read the ReleaseNotes of all versions between your current version
and the version to which you are upgrading. If the Bacula catalog database has
been upgraded, you will either need to reinitialize your database starting
from scratch, or save an ASCII copy of your database, then proceed to upgrade
it. If there are several database upgrades between your version and the
version to which you are upgradding, you will need to apply each database
upgrade script. For your convenience, you can find all the old upgrade scripts
in the {\bf upgradedb} directory of the source code. You will need to edit the
scripts to correspond to your system configuration. The final upgrade script,
if any, will be in the {\bf src/cats} directory as described in the
ReleaseNotes. 

If you are upgrading from one major version to another, you will need to
replace all your components at the same time as generally the inter-daemon
protocol will change. However, within any particular release (e.g. version
1.32.x) unless there is an oversight or bug, the daemon protocol will not
change. If this is confusing, simply read the ReleaseNotes very carefully as
they will note if all daemons must be upgraded at the same time. 

Finally, please note that in general it is not necessary to do a 
{\bf make uninstall} before doing an upgrade. In fact, if you do so, you will 
most likely delete all your conf files, which could be disasterous.
For additional information on upgrading, please see the \ilink{Upgrading Bacula
Versions}{upgrading} in the Tips chapter of this manual.


\subsection*{Dependency Packages}
\label{Dependency}
\index[general]{Dependency Packages }
\index[general]{Packages!Dependency }
\addcontentsline{toc}{subsection}{Dependency Packages}

As discussed above, we have combined a number of third party packages that
Bacula might need into the {\bf depkgs} and {\bf depkgs1} releases. You can,
of course, get the latest packages from the original authors. The locations of
where we obtained the packages are in the README file in each package.
However, be aware that the packages in the depkgs files have been tested by us
for compatibility with Bacula. 

Typically, a dependency package will be named {\bf depkgs-ddMMMyy.tar.gz} and
{\bf depkgs1-ddMMyy.tar.gz} where {\bf dd} is the day we release it, {\bf MMM}
is the abbreviated month (e.g. Jan), and {\bf yy} is the year. An actual
example is: {\bf depkgs-07Apr02.tar.gz}. To install and build this package (if
needed), you do the following: 

\begin{enumerate}
\item Create a {\bf bacula} directory, into which you will place  both the
   Bacula source as well as the dependency package.  
\item Detar the {\bf depkg} into the {\bf bacula} directory.  
\item cd bacula/depkgs  
\item make 
   \end{enumerate}

Although the exact composition of the dependency packages may change from time
to time, the current makeup is the following: 

\addcontentsline{lot}{table}{Depedency Packages}
\begin{longtable}{|l|l|l|l|}
 \hline 
\multicolumn{1}{|c| }{\bf 3rd Party Package } & \multicolumn{1}{c| }{\bf
depkgs } & \multicolumn{1}{c| }{\bf depkgs1 } & \multicolumn{1}{c| }{\bf
depkgs-win32  } \\
 \hline {SQLite } & \multicolumn{1}{c| }{X } & \multicolumn{1}{c| }{- } &
\multicolumn{1}{c| }{-  } \\
 \hline {mtx } & \multicolumn{1}{c| }{X } & \multicolumn{1}{c| }{- } &
\multicolumn{1}{c| }{-  } \\
 \hline {readline } & \multicolumn{1}{c| }{- } & \multicolumn{1}{c| }{X } &
\multicolumn{1}{c| }{-  } \\
 \hline {pthreads } & \multicolumn{1}{c| }{- } & \multicolumn{1}{c| }{- } &
\multicolumn{1}{c| }{X  } \\
 \hline {zlib } & \multicolumn{1}{c| }{- } & \multicolumn{1}{c| }{- } &
\multicolumn{1}{c| }{X  } \\
 \hline {wxWidgits } & \multicolumn{1}{c| }{- } & \multicolumn{1}{c| }{- } &
\multicolumn{1}{c| }{X }
\\ \hline 

\end{longtable}

Note, some of these packages are quite large, so that building them can be a
bit time consuming. The above instructions will build all the packages
contained in the directory. However, when building Bacula, it will take only
those pieces that it actually needs. 

Alternatively, you can make just the packages that are needed. For example, 

\footnotesize
\begin{verbatim}
cd bacula/depkgs
make sqlite
\end{verbatim}
\normalsize

will configure and build only the SQLite package. 

You should build the packages that you will require in {\bf depkgs} and/or
{\bf depkgs1} prior to configuring and building Bacula, since Bacula will need
them during the build process. 

Even if you do not use SQLite, you might find it worthwhile to build {\bf mtx}
because the {\bf tapeinfo} program that comes with it can often provide you
with valuable information about your SCSI tape drive (e.g. compression,
min/max block sizes, ...). 

The {\bf depkgs-win32} package contains the source code for the pthreads and
zlib libraries used by the native Win32 client program. It will only be needed
if you intend to build the Win32 client from source. 

\subsection*{Supported Operating Systems}
\label{Systems}
\index[general]{Systems!Supported Operating }
\index[general]{Supported Operating Systems }
\addcontentsline{toc}{subsection}{Supported Operating Systems}

Please see the 
\ilink{ Supported Operating Systems}{SupportedOSes} section
of the QuickStart chapter of this manual. 

\subsection*{Building Bacula from Source}
\label{Building}
\index[general]{Source!Building Bacula from }
\index[general]{Building Bacula from Source }
\addcontentsline{toc}{subsection}{Building Bacula from Source}

The basic installation is rather simple. 

\begin{enumerate}
\item Install and build any {\bf depkgs} as noted above.  
\item Configure and install MySQL or PostgreSQL (if desired). 
   \ilink{Installing and Configuring MySQL Phase I}{_ChapterStart} or  
   \ilink{Installing and Configuring PostgreSQL Phase
   I}{_ChapterStart10}.  If you are installing from rpms, and are
   using MySQL, please be sure to install  {\bf mysql-devel}, so that the MySQL
   header files are available  while compiling Bacula. In addition, the MySQL
   client  library {\bf mysqlclient} requires the gzip compression library  {\bf
   libz.a} or {\bf libz.so}. If you are using rpm packages,  these libraries are
   in the {\bf libz-devel} package. On Debian  systems, you will need to load the
   {\bf zlib1g-dev} package. If  you are not using rpms or debs, you will need to
   find the  appropriate package for your system.  
   Note, if you already have a running MySQL or PostgreSQL on your system, you 
   can skip this phase provided that you have built the thread  safe libraries.
   And you have already installed the additional  rpms noted above.  
\item As an alternative to MySQL and PostgreSQL, configure and install SQLite,
    which is part of the {\bf depkgs}.  
   \ilink{Installing and Configuring SQLite}{_ChapterStart33}.  
   \item Detar the Bacula source code preferably into the {\bf bacula}  directory
   discussed above.  
\item {\bf cd} to the directory containing the source code.  
\item ./configure (with appropriate options as described below)  
\item Check the output of ./configure very carefully, especially  the Install
   binaries and Install config files directories.  If they are not correct,
   please rerun ./configure until they  are. The output from ./configure is
   stored in {\bf config.out}  and can be re-displayed at any time without
   rerunning the  ./configure by doing {\bf cat config.out}.  
\item If after running ./configure once, you decide to change options  and
   re-run it, that is perfectly fine, but before re-running it,  you should run: 


\footnotesize
\begin{verbatim}
      make distclean
      
\end{verbatim}
\normalsize

so that you are sure to start from scratch and not have a  mixture of the two
options. This is because ./configure  caches much of the information. The {\bf
make distclean}  is also critical if you move the source file from one 
machine to another. If the {\bf make distclean} fails,  just ignore it and
continue on.  
\item make  

   If you get errors while linking in the Storage daemon  directory (src/stored),
it is probably because you have not  loaded the static libraries on your
system. I noticed this  problem on a Solaris system. To correct it, make sure
that you  have not added {\bf \verb{--{enable-static-tools} to the  {\bf ./configure}
command. 
\item make install  
\item If you are new to Bacula, we {\bf strongly} recommend that you  skip the
   next step and use the default configuration files,  then run the example
   program in the next chapter, then  come back and modify your configuration
files to suit your  particular needs.  
\item Customize the configuration files for each of the three daemons 
   (Directory, File, Storage) and for the Console program. For the  details of
   how to do this, please see 
\ilink{Setting Up Bacula Configuration Files}{_ChapterStart16} in
the Configuration chapter of this manual.  We recommend that you start by
modifying the default configuration  files supplied, making the minimum
changes necessary.  Complete customization can be done after you have Bacula 
up and running.  Please take care when modifying passwords, which  were
randomly generated, and the {\bf Name}s as the  passwords and names must agree
between the configuration files  for security reasons. 
\item Create the Bacula MySQL database and tables (if using MySQL)  
   \ilink{Installing and Configuring MySQL Phase II}{mysql_phase2} or 
   create the Bacula PostgreSQL database and tables  
\ilink{Installing and Configuring PostgreSQL Phase
II}{PostgreSQL_phase2} or alternatively  if you are using
SQLite  
\ilink{Installing and Configuring SQLite Phase II}{phase2}.  
\item Start Bacula ({\bf ./bacula start}) Note. the next chapter  shows you
   how to do this in detail.  
\item Interface with Bacula using the Console program  
\item For the previous two items, please follow the instructions  in the 
   \ilink{Running Bacula}{_ChapterStart1} chapter of  this manual,
   where you will run a simple backup and do a  restore. Do this before you make
heavy modifications to the  configuration files so that you are sure that
Bacula works  and are familiar with it. After that changing the conf files 
will be easier.  
\item If after installing Bacula, you decide to ``move it'', that is  to
   install it in a different set of directories, proceed  as follows:  

\footnotesize
\begin{verbatim}
      make uninstall
      make distclean
      ./configure (your-new-options)
      make
      make install
      
\end{verbatim}
\normalsize

\end{enumerate}

If all goes well, the {\bf ./configure} will correctly determine which
operating system you are running and configure the source code appropriately.
Currently, FreeBSD, Linux (RedHat), and Solaris are supported. MacOS X 10.3 is
reported to work with the Client only as long as readline support is disabled.


If you install Bacula on more than one system, and they are identical, you can
simply transfer the source tree to that other system and do a ``make
install''. However, if there are differences in the libraries or OS versions,
or you wish to install on a different OS, you should start from the original
compress tar file. If you do transfer the source tree, and you have previously
done a ./configure command, you MUST do: 

\footnotesize
\begin{verbatim}
make distclean
\end{verbatim}
\normalsize

prior to doing your new ./configure. This is because the GNU autoconf tools
cache the configuration, and if you re-use a configuration for a Linux machine
on a Solaris, you can be sure your build will fail. To avoid this, as
mentioned above, either start from the tar file, or do a ``make distclean''. 

In general, you will probably want to supply a more complicated {\bf
configure} statement to ensure that the modules you want are built and that
everything is placed into the correct directories. 

For example, on RedHat, one could use the following: 

\footnotesize
\begin{verbatim}
CFLAGS="-g -Wall" \
  ./configure \
    --sbindir=$HOME/bacula/bin \
    --sysconfdir=$HOME/bacula/bin \
    --with-pid-dir=$HOME/bacula/bin/working \
    --with-subsys-dir=$HOME/bacula/bin/working \
    --with-mysql=$HOME/mysql \
    --with-working-dir=$HOME/bacula/bin/working \
    --with-dump-email=$USER
\end{verbatim}
\normalsize

Note, the advantage of using the above configuration to start is that
everything will be put into a single directory, which you can later delete
once you have run the examples in the next chapter and learned how Bacula
works. In addition, the above can be installed and run as non-root. 

For the developer's convenience, I have added a {\bf defaultconfig} script to
the {\bf examples} directory. This script contains the statements that you
would normally use, and each developer/user may modify them to suit his needs.
You should find additional useful examples in this directory as well. 

The {\bf \verb{--{enable-conio} or {\bf \verb{--{enable-readline} options are useful because
they provide a command line history and editing capability for the Console
program. If you have included either option in the build, either the {\bf
termcap} or the {\bf ncurses} package will be needed to link. On some systems,
such as SuSE, the termcap library is not in the standard library directory. As
a consequence, the option may be disabled or you may get an error message such
as: 

\footnotesize
\begin{verbatim}
/usr/lib/gcc-lib/i586-suse-linux/3.3.1/.../ld:
cannot find -ltermcap
collect2: ld returned 1 exit status
\end{verbatim}
\normalsize

while building the Bacula Console. In that case, you will need to set the {\bf
LDFLAGS} environment variable prior to building. 

\footnotesize
\begin{verbatim}
export LDFLAGS="-L/usr/lib/termcap"
\end{verbatim}
\normalsize

The same library requirements apply if you wish to use the readline
subroutines for command line editing and history or
 if you are using a MySQL library that requires encryption. If you need encryption,
you can either export the appropriate additional library options as shown
above or, alternatively, you can include them directly on the ./configure line
as in: 

\footnotesize
\begin{verbatim}
LDFLAGS="-lssl -lcyrpto" \
   ./configure \
      <your-options>
\end{verbatim}
\normalsize

On some systems such as Mandrake, readline tends to
gobble up prompts, which makes it totally useless. If this happens to you, use
the disable option, or if you are using version 1.33 and above try using {\bf
\verb{--{enable-conio} to use a built-in readline replacement. You will still need
the either termcap or ncurses library, but it is unlikely that the {\bf conio}
package will gobble up prompts. 

readline is no longer supported after version 1.34. The code is still
available and if users submit patches for it, I will be happy to apply them.
However, due to the fact that each version of readline seems to be
incompatible with previous versions, and that there are significant
differences between systems, I can no longer afford to support it. 

\subsection*{What Database to Use?}
\label{DB}
\index[general]{What Database to Use? }
\index[general]{Use!What Database to }
\addcontentsline{toc}{subsection}{What Database to Use?}

Before building Bacula you need to decide if you want to use SQLite, MySQL, or
PostgreSQL. If you are not already running MySQL or PostgreSQL, we recommend
that you start by using SQLite. This will greatly simplify the setup for you
because SQLite is compiled into Bacula an requires no administration. It
performs well and is suitable for small to medium sized installations (maximum
10-20 machines). 

If you wish to use MySQL as the Bacula catalog, please see the 
\ilink{Installing and Configuring MySQL}{_ChapterStart} chapter of
this manual. You will need to install MySQL prior to continuing with the
configuration of Bacula. MySQL is a high quality database that is very
efficient and is suitable for any sized installation. It is slightly more
complicated than SQLite to setup and administer because it has a number of
sophisticated features such as userids and passwords. It runs as a separate
process, is truly professional and can manage a database of any size. 

If you wish to use PostgreSQL as the Bacula catalog, please see the 
\ilink{Installing and Configuring PostgreSQL}{_ChapterStart10}
chapter of this manual. You will need to install PostgreSQL prior to
continuing with the configuration of Bacula. PostgreSQL is very similar to
MySQL, though it tends to be slightly more SQL92 compliant and has many more
advanced features such as transactions, stored procedures, and the such. It
requires a certain knowledge to install and maintain. There are some important
performance problems with PostgreSQL in Bacula versions prior to 1.35.5. 

If you wish to use SQLite as the Bacula catalog, please see 
\ilink{Installing and Configuring SQLite}{_ChapterStart33} chapter of
this manual. 

\subsection*{Quick Start}
\index[general]{Quick Start }
\index[general]{Start!Quick }
\addcontentsline{toc}{subsection}{Quick Start}

There are a number of options and important considerations given below
that you can skip for the moment if you have not had any problems building
Bacula with a simplified configuration as shown above. 

If you want to dive right into it, we recommend you skip to the next chapter,
and run the example program. It will teach you a lot about Bacula and as an
example can be installed into a single directory (for easy removal) and run as
non-root. If you have any problems or when you want to do a real installation,
come back to this chapter and read the details presented below. 

\subsection*{Configure Options}
\label{Options}
\index[general]{Options!Configure }
\index[general]{Configure Options }
\addcontentsline{toc}{subsection}{Configure Options}

The following command line options are available for {\bf configure} to
customize your installation. 

\begin{description}

\item [ \verb{--{sysbindir=\lt{}binary-path\gt{}]
   \index[dir]{\verb{--{sysbindir }
   Defines where the Bacula  binary (executable) files will be placed during a
   {\bf make  install} command.  

\item [ \verb{--{sysconfdir=\lt{}config-path\gt{}]
   \index[dir]{\verb{--{sysconfdir }
   Defines where the  Bacula configuration files should be placed during a {\bf
   make  install} command.  

\item [ \verb{--{enable-smartalloc ]
   \index[dir]{\verb{--{enable-smartalloc }
   This enables the inclusion of the  Smartalloc orphaned buffer detection code.
   This option is highly  recommended. Because we never build without this
   option, you may  experience problems if it is not enabled. In this case,
   simply  re-enable the option. We strongly recommend keeping this option 
   enabled as it helps detect memory leaks. This configuration  parameter is used
   while building Bacula  

\item [ \verb{--{enable-gnome ]
   \index[dir]{\verb{--{enable-gnome }
   If you have GNOME installed on your  computer and you want to use the GNOME
   GUI Console interface  to Bacula, you must specify this option. Doing so  will
   build everything in the {\bf src/gnome-console} directory.  

\item [ \verb{--{enable-wx-console ]
   \index[console]{\verb{--{enable-wx-console }
   If you have wxWidgets installed on your  computer and you want to use the
   wxWidgets GUI Console interface  to Bacula, you must specify this option.
   Doing so  will build everything in the {\bf src/wx-console} directory. This 
   could also be useful to users who want a GUI Console and don't  want to
   install Gnome, as wxWidgets can work with GTK+, Motif or even X11  libraries. 


\item [ \verb{--{enable-tray-monitor ]
   \index[dir]{\verb{--{enable-tray-monitor }
   If you have GTK installed on your  computer, you run an graphical environment
   or a window manager compatible  with the FreeDesktop system tray standard
   (like KDE and GNOME)  and you want to use a GUI to monitor Bacula daemons, you
   must specify  this option. Doing so will build everything in the  {\bf
   src/tray-monitor} directory.  

\item [ \verb{--{enable-static-tools]
   \index[dir]{\verb{--{enable-static-tools }
   This option causes the linker  to link the Storage daemon utility tools ({\bf
   bls},  {\bf bextract}, and {\bf bscan}) statically. This permits using  them
   without having the shared libraries loaded. If you have  problems linking in
   the {\bf src/stored} directory, make sure  you have not enabled this option,
   or explicitly  disable static linking by adding {\bf \verb{--{disable-static-tools}. 


\item [ \verb{--{enable-static-fd]
   \index[fd]{\verb{--{enable-static-fd }
   This option causes the make process  to build a {\bf static-bacula-fd} in
   addition to the standard  File daemon. This static version will include
   statically linked  libraries and is required for the Bare Metal recovery. This
   option  is largely superseded by using {\bf make static-bacula-fd} from with 
   in the {\bf src/filed} directory. Also, the {\bf \verb{--{enable-client-only}  option
   described below is useful for just building a client so that  all the other
   parts of the program are not compiled.  

\item [ \verb{--{enable-static-sd]
   \index[sd]{\verb{--{enable-static-sd }
   This option causes the make process  to build a {\bf static-bacula-sd} in
   addition to the standard  Storage daemon. This static version will include
   statically linked  libraries and could be useful during a Bare Metal recovery.
 

\item [ \verb{--{enable-static-dir]
   \index[dir]{\verb{--{enable-static-dir }
   This option causes the make process  to build a {\bf static-bacula-dir} in
   addition to the standard  Director. This static version will include
   statically linked  libraries and could be useful during a Bare Metal recovery.


\item [ \verb{--{enable-static-cons]
   \index[dir]{\verb{--{enable-static-cons }
   This option causes the make process  to build a {\bf static-console} and a
   {\bf static-gnome-console}  in addition to the standard console.  This static
   version will include statically linked  libraries and could be useful during a
   Bare Metal recovery.  

\item [ \verb{--{enable-client-only]
   \index[console]{\verb{--{enable-client-only }
   This option causes the make process  to build only the File daemon and the
   libraries that it needs.  None of the other daemons, storage tools, nor the
   console will  be built. Likewise a {\bf make install} will then only install 
   the File daemon. To cause all daemons to be built, you will need  to do a
   configuration without this option. This option greatly  facilitates building a
   Client on a client only machine.  

\item [ \verb{--{enable-largefile]
   \index[console]{\verb{--{enable-largefile }
   This option (default) causes  Bacula to be built with 64 bit file address
   support if it  is available on your system. This permits Bacula to read and 
   write files greater than 2 GBytes in size. You may disable this  feature and
   revert to 32 bit file addresses by using  {\bf \verb{--{disable-largefile}.  

\item [ \verb{--{with-sqlite=\lt{}sqlite-path\gt{}]
   \index[fd]{\verb{--{with-sqlite }
   This enables use of  the SQLite database. The {\bf sqlite-path} is not
   normally  specified as Bacula looks for the necessary components in  a
   standard location ({\bf depkgs/sqlite}). See 
   \ilink{Installing and Configuring SQLite}{_ChapterStart33} chapter of
    this manual for more details.  

\item [ \verb{--{with-mysql=\lt{}mysql-path\gt{}]
   \index[fd]{\verb{--{with-mysql }
   This enables building of the Catalog services for Bacula. It  assumes that
   MySQL is running on your system, and expects  it to be installed in the {\bf
   mysql-path} that you  specify. If this option is not present, the build will 
   automatically include the internal Bacula database  code. We recommend that
   you use this option if possible.  If you do use this option, please proceed to
   installing  MySQL in the 
   \ilink{Installing and Configuring MySQL}{_ChapterStart} chapter
   before proceeding with the configuration.  

\item [ \verb{--{with-postgresql=\lt{}path\gt{}]
   \index[fd]{\verb{--{with-postgresql }
   This provides an explicit  path to the PostgreSQL libraries if Bacula cannot
   find it by  default.  

\item [ \verb{--{enable-conio]
   \index[fd]{\verb{--{enable-conio }
   Tells Bacula to enable building the  small, light weight readline replacement
   routine. It is generally  much easier to configure than readline, although,
   like readline,  it needs either the termcap or ncurses library.  

\item [ \verb{--{with-readline=\lt{}readline-path\gt{}]
   \index[fd]{\verb{--{with-readline }
   Tells Bacula  where {\bf readline} is installed. Normally, Bacula will  find
   readline if it is in a standard library. If it is not found  and no
   \verb{--{with-readline is specified, readline will be disabled.  This option affects
   the Bacula build. Readline provides  the Console program with a command line
   history and editing  capability and is no longer supports, so you are on your
   own  if you have problems. 

\item [ \verb{--{enable-readline]
   \index[fd]{\verb{--{enable-readline }
   Tells Bacula to enable readline support.  It is normally disabled due to the
   large number of configuration  problems and the fact that the package seems to
   change in incompatible  ways from version to version.  

\item [ \verb{--{with-tcp-wrappers=\lt{}path\gt{}]
   \index[fd]{\verb{--{with-tcp-wrappers }
   This specifies that you  want TCP wrappers (man hosts\_access(5)) compiled in.
   The path is optional since  Bacula will normally find the libraries in the
   standard locations.  This option affects the Bacula build.  In specifying your
   restrictions in the {\bf /etc/hosts.allow}  or {\bf /etc/hosts.deny} files, do
   not use the {\bf twist}  option (hosts\_options(5)) or the Bacula process will
   be terminated.  
   
   For more information on configuring and testing TCP wrappers, please  see the 
   \ilink{Configuring and Testing TCP Wrappers}{wrappers}  section
   in the Security Capter.  

\item [ \verb{--{with-working-dir=\lt{}working-directory-path\gt{} ]
   \index[dir]{\verb{--{with-working-dir }
   This option is mandatory and specifies a directory  into which Bacula may
   safely place files that  will remain between Bacula executions. For example, 
   if the internal database is used, Bacula will keep  those files in this
   directory.  This option is only used to modify the daemon  configuration
   files. You may also accomplish the same  thing by directly editing them later.
   The working directory  is not automatically created by the install process, so
   you  must ensure that it exists before using Bacula for the  first time. 

\item [ \verb{--{with-base-port=\lt{}port=number\gt{}]
   \index[dir]{\verb{--{with-base-port }
   In order to run,  Bacula needs three TCP/IP ports (one for the Bacula 
   Console, one for the Storage daemon, and one for the File daemon).  The {\bf
   \verb{--{with-baseport} option will automatically assign three  ports beginning at
   the base port address specified. You may  also change the port number in the
   resulting configuration  files. However, you need to take care that the
   numbers  correspond correctly in each of the three daemon configuration 
   files. The default base port is 9101, which assigns ports 9101  through 9103.
   These ports (9101, 9102, and 9103) have been  officially assigned to Bacula by
   IANA.  This option is only used  to modify the daemon configuration files. You
   may also accomplish  the same thing by directly editing them later. 

\item [ \verb{--{with-dump-email=\lt{}email-address\gt{}]
   \index[dir]{\verb{--{with-dump-email }
   This option specifies  the email address where any core dumps should be set.
   This option  is normally only used by developers.  

\item [ \verb{--{with-pid-dir=\lt{}PATH\gt{}  ]
   \index[dir]{\verb{--{with-pid-dir }
   This specifies where Bacula should place the process id  file during
   execution. The default is: {\bf /var/run}.  This directory is not created by
   the install process, so  you must ensure that it exists before using Bacula
   the  first time.  

\item [ \verb{--{with-subsys-dir=\lt{}PATH\gt{}]
   \index[dir]{\verb{--{with-subsys-dir }
   This specifies where Bacula should place the subsystem lock  file during
   execution. The default is {\bf /var/run/subsys}.  Please make sure that you do
   not specify the same directory  for this directory and for the {\bf sbindir}
   directory.  This directory is used only within the autostart scripts.  The
   subsys directory is not created by the Bacula install,  so you must be sure to
   create it before using Bacula. 

\item [ \verb{--{with-dir-password=\lt{}Password\gt{}]
   \index[dir]{\verb{--{with-dir-password }
   This option allows you to specify the password used to  access the Directory
   (normally from the Console program).  If it is not specified, configure will
   automatically create a random  password.  

\item [ \verb{--{with-fd-password=\lt{}Password\gt{} ]
   \index[fd]{\verb{--{with-fd-password }
   This option allows you to specify the password used to  access the File daemon
   (normally called from the Director).  If it is not specified, configure will
   automatically create a random  password.  

\item [ \verb{--{with-sd-password=\lt{}Password\gt{} ]
   \index[sd]{\verb{--{with-sd-password }
   This option allows you to specify the password used to  access the Directory
   (normally called from the Director).  If it is not specified, configure will
   automatically create a random  password.  

\item [ \verb{--{with-dir-user=\lt{}User\gt{} ]
   \index[dir]{\verb{--{with-dir-user }
   This option allows you to specify the Userid used to  run the Director. The
   Director must be started as root, but  doesn't need to run as root, and  after
   doing preliminary initializations, it can ``drop''  to the UserId specified on
   this option. 

\item [ \verb{--{with-dir-group=\lt{}Group\gt{} ]
   \index[dir]{\verb{--{with-dir-group }
   This option allows you to specify the GroupId used to  run the Director. The
   Director must be started as root, but  doesn't need to run as root, and  after
   doing preliminary initializations, it can ``drop''  to the GroupId specified
   on this option. 

\item [ \verb{--{with-sd-user=\lt{}User\gt{} ]
   \index[sd]{\verb{--{with-sd-user }
   This option allows you to specify the Userid used to  run the Storage daemon.
   The Storage daemon must be started as root, but  doesn't need to run as root,
   and  after doing preliminary initializations, it can ``drop''  to the UserId
   specified on this option. If you use this option,  you will need to take care
   that the Storage daemon has access  to all the devices (tape drives, ...) that
   it needs. 

\item [ \verb{--{with-sd-group=\lt{}Group\gt{} ]
   \index[sd]{\verb{--{with-sd-group }
   This option allows you to specify the GroupId used to  run the Storage daemon.
   The Storage daemon must be started as root, but  doesn't need to run as root,
   and  after doing preliminary initializations, it can ``drop''  to the GroupId
   specified on this option. 

\item [ \verb{--{with-fd-user=\lt{}User\gt{} ]
   \index[fd]{\verb{--{with-fd-user }
   This option allows you to specify the Userid used to  run the File daemon. The
   File daemon must be started as root,  and in most cases, it needs to run as
   root, so this option is  used only in very special cases,  after doing
   preliminary initializations, it can ``drop''  to the UserId specified on this
   option. 

\item [ \verb{--{with-fd-group=\lt{}Group\gt{} ]
   \index[fd]{\verb{--{with-fd-group }
   This option allows you to specify the GroupId used to  run the File daemon.
   The File daemon must be started as root, and  in most cases, it must be run as
   root, however,  after doing preliminary initializations, it can ``drop''  to
   the GroupId specified on this option. 

\end{description}

Note, many other options are presented when you do a {\bf ./configure \verb{--{help},
but they are not implemented. 

\subsection*{Recommended Options for most Systems}
\index[general]{Systems!Recommended Options for most }
\index[general]{Recommended Options for most Systems }
\addcontentsline{toc}{subsection}{Recommended Options for most Systems}

For most systems, we recommend starting with the following options: 

\footnotesize
\begin{verbatim}
./configure \
  --enable-smartalloc \
  --sbindir=$HOME/bacula/bin \
  --sysconfdir=$HOME/bacula/bin \
  --with-pid-dir=$HOME/bacula/bin/working \
  --with-subsys-dir=$HOME/bacula/bin/working \
  --with-mysql=$HOME/mysql \
  --with-working-dir=$HOME/bacula/working
\end{verbatim}
\normalsize

If you want to install Bacula in an installation directory rather than run it
out of the build directory (as developers will do most of the time), you
should also include the \verb{--{sbindir and \verb{--{sysconfdir options with appropriate
paths. Neither are necessary if you do not use ``make install'' as is the case
for most development work. The install process will create the sbindir and
sysconfdir if they do not exist, but it will not automatically create the
pid-dir, subsys-dir, or working-dir, so you must ensure that they exist before
running Bacula for the first time. See below for an example of how Kern does
it. 

\subsection*{RedHat}
\index[general]{RedHat }
\addcontentsline{toc}{subsection}{RedHat}

Using SQLite: 

\footnotesize
\begin{verbatim}
 
CFLAGS="-g -Wall" ./configure \
  --sbindir=$HOME/bacula/bin \
  --sysconfdir=$HOME/bacula/bin \
  --enable-smartalloc \
  --with-sqlite=$HOME/bacula/depkgs/sqlite \
  --with-working-dir=$HOME/bacula/working \
  --with-pid-dir=$HOME/bacula/bin/working \
  --with-subsys-dir=$HOME/bacula/bin/working \
  --enable-gnome \
  --enable-conio
\end{verbatim}
\normalsize

or 

\footnotesize
\begin{verbatim}
 
CFLAGS="-g -Wall" ./configure \
  --sbindir=$HOME/bacula/bin \
  --sysconfdir=$HOME/bacula/bin \
  --enable-smartalloc \
  --with-mysql=$HOME/mysql \
  --with-working-dir=$HOME/bacula/working
  --with-pid-dir=$HOME/bacula/bin/working \
  --with-subsys-dir=$HOME/bacula/bin/working
  --enable-gnome \
  --enable-conio
\end{verbatim}
\normalsize

or finally, a completely traditional RedHat Linux install: 

\footnotesize
\begin{verbatim}
CFLAGS="-g -Wall" ./configure \
  --prefix=/usr \
  --sbindir=/usr/sbin \
  --sysconfdir=/etc/bacula \
  --with-scriptdir=/etc/bacula \
  --enable-smartalloc \
  --enable-gnome \
  --with-mysql \
  --with-working-dir=/var/bacula \
  --with-pid-dir=/var/run \
  --with-subsys-dir=/var/lock/subsys \
  --enable-conio
\end{verbatim}
\normalsize

Note, Bacula assumes that /var/bacula, /var/run, and /var/loc/subsys exist so
it will not automatically create them during the install process. 

\subsection*{Solaris}
\index[general]{Solaris }
\addcontentsline{toc}{subsection}{Solaris}

To build Bacula from source, you will need the following installed on your
system (they are not by default): libiconv, gcc 3.3.2, stdc++, libgcc (for
stdc++ and gcc\_s libraries), make 3.8 or later. 

You will probably also need to: Add /usr/local/bin to PATH and Add
/usr/ccs/bin to PATH for ar. 

\footnotesize
\begin{verbatim}
#!/bin/sh
CFLAGS="-g" ./configure \
  --sbindir=$HOME/bacula/bin \
  --sysconfdir=$HOME/bacula/bin \
  --with-mysql=$HOME/mysql \
  --enable-smartalloc \
  --with-pid-dir=$HOME/bacula/bin/working \
  --with-subsys-dir=$HOME/bacula/bin/working \
  --with-working-dir=$HOME/bacula/working
\end{verbatim}
\normalsize

As mentioned above, the install process will create the sbindir and sysconfdir
if they do not exist, but it will not automatically create the pid-dir,
subsys-dir, or working-dir, so you must ensure that they exist before running
Bacula for the first 

\subsection*{FreeBSD}
\index[general]{FreeBSD }
\addcontentsline{toc}{subsection}{FreeBSD}

Please see: 
\elink{The FreeBSD Diary}{http://www.freebsddiary.org/bacula.php} for a
detailed description on how to make Bacula work on your system. In addition,
users of FreeBSD prior to 4.9-STABLE dated Mon Dec 29 15:18:01 2003 UTC who
plan to use tape devices, please see the 
\ilink{Tape Testing Chapter}{FreeBSDTapes} of this manual for
{\bf important} information on how to configure your tape drive for
compatibility with Bacula. 

If you are using Bacula with MySQL, you should take care to compile MySQL with
FreeBSD native threads rather than LinuxThreads, since Bacula is normal built
with FreeBSD native threads rather than LinuxTreads. Mixing the two will
probably not work. 

\subsection*{Win32}
\index[general]{Win32 }
\addcontentsline{toc}{subsection}{Win32}

To install the binary Win32 version of the File daemon please see the 
\ilink{Win32 Installation Chapter}{_ChapterStart7} in this document. 

\subsection*{Windows Systems with CYGWIN Installed}
\label{Win32}
\index[general]{Windows Systems with CYGWIN Installed }
\index[general]{Installed!Windows Systems with CYGWIN }
\addcontentsline{toc}{subsection}{Windows Systems with CYGWIN Installed}

As of version 1.34, Bacula no longer uses CYGWIN for the Win32 File daemon.
However, it is still built under a CYGWIN build environment -- though you can
probably do it with VC Studio only. If you wish to build the Win32 File daemon
from the source, you will need Microsoft C++ version 6.0 or greater. In Bacula
prior to version 1.33, CYGWIN was used. Details for building it are in the
README file of the src/win32 directory. 

Note, although most parts of Bacula build on Windows systems, the only part
that we have tested and used is the File daemon. 

Finally, you should follow the installation instructions in the 
\ilink{Win32 Installation}{_ChapterStart7} section of this document. 

\subsection*{Kern's Configure Script}
\index[general]{Script!Kern's Configure }
\index[general]{Kern's Configure Script }
\addcontentsline{toc}{subsection}{Kern's Configure Script}

The script that I use for building on my ``production'' Linux machines is: 

\footnotesize
\begin{verbatim}
#!/bin/sh
# This is Kern's configure script for Bacula
CFLAGS="-g -Wall" \
  ./configure \
    --sbindir=$HOME/bacula/bin \
    --sysconfdir=$HOME/bacula/bin \
    --enable-smartalloc \
    --enable-gnome \
    --with-pid-dir=$HOME/bacula/bin/working \
    --with-subsys-dir=$HOME/bacula/bin/working \
    --with-mysql=$HOME/mysql \
    --with-working-dir=$HOME/bacula/bin/working \
    --with-dump-email=$USER \
    --with-smtp-host=mail.your-site.com \
    --with-baseport=9101
exit 0
\end{verbatim}
\normalsize

Note that I define the base port as 9101, which means that Bacula will use
port 9101 for the Director console, port 9102 for the File daemons, and port
9103 for the Storage daemons. These ports should be available on all systems
because they have been officially assigned to Bacula by IANA (Internet
Assigned Numbers Authority). We strongly recommend that you use only these
ports to prevent any conflicts with other programs. This is in fact the
default if you do not specify a {\bf \verb{--{with-baseport} option. 

You may also want to put the following entries in your {\bf /etc/services}
file as it will make viewing the connections made by Bacula easier to
recognize (i.e. netstat -a): 

\footnotesize
\begin{verbatim}
bacula-dir      9101/tcp
bacula-fd       9102/tcp
bacula-sd       9103/tcp
\end{verbatim}
\normalsize

\subsection*{Installing Bacula}
\index[general]{Bacula!Installing }
\index[general]{Installing Bacula }
\addcontentsline{toc}{subsection}{Installing Bacula}

Before setting up your configuration files, you will want to install Bacula in
its final location. Simply enter: 

\footnotesize
\begin{verbatim}
make install
\end{verbatim}
\normalsize

If you have previously installed Bacula, the old binaries will be overwritten,
but the old configuration files will remain unchanged, and the ``new''
configuration files will be appended with a {\bf .new}. Generally if you have
previously installed and run Bacula you will want to discard or ignore the
configuration files with the appended {\bf .new}. 

\subsection*{Building a File Daemon or Client}
\index[general]{Client!Building a File Daemon or }
\index[general]{Building a File Daemon or Client }
\addcontentsline{toc}{subsection}{Building a File Daemon or Client}

If you run the Director and the Storage daemon on one machine and you wish to
back up another machine, you must have a copy of the File daemon for that
machine. If the machine and the Operating System are identical, you can simply
copy the Bacula File daemon binary file {\bf bacula-fd} as well as its
configuration file {\bf bacula-fd.conf} then modify the name and password in
the conf file to be unique. Be sure to make corresponding additions to the
Director's configuration file ({\bf bacula-dir.conf}). 

If the architecture or the O/S level are different, you will need to build a
File daemon on the Client machine. To do so, you can use the same {\bf
./configure} command as you did for your main program, starting either from a
fresh copy of the source tree, or using {\bf make\ distclean} before the {\bf
./configure}. 

Since the File daemon does not access the Catalog database, you can remove the
{\bf \verb{--{with-mysql} or {\bf \verb{--{with-sqlite} options, then add {\bf
\verb{--{enable-client-only}. This will compile only the necessary libraries and the
client programs and thus avoids the necessity of installing one or another of
those database programs to build the File daemon. With the above option, you
simply enter {\bf make} and just the client will be built. 
\label{autostart}

\subsection*{Auto Starting the Daemons}
\index[general]{Daemons!Auto Starting the }
\index[general]{Auto Starting the Daemons }
\addcontentsline{toc}{subsection}{Auto Starting the Daemons}

If you wish the daemons to be automatically started and stopped when your
system is booted (a good idea), one more step is necessary. First, the
./configure process must recognize your system -- that is it must be a
supported platform and not {\bf unknown}, then you must install the platform
dependent files by doing: 

\footnotesize
\begin{verbatim}
(become root)
make install-autostart
\end{verbatim}
\normalsize

Please note, that the auto-start feature is implemented only on systems that
we officially support (currently, FreeBSD, RedHat Linux, and Solaris), and has
only been fully tested on RedHat Linux. 

The {\bf make install-autostart} will cause the appropriate startup scripts to
be installed with the necessary symbolic links. On RedHat Linux systems, these
scripts reside in {\bf /etc/rc.d/init.d/bacula-dir} {\bf
/etc/rc.d/init.d/bacula-fd}, and {\bf /etc/rc.d/init.d/bacula-sd}. However the
exact location depends on what operating system you are using. 

If you only wish to install the File daemon, you may do so with: 

\footnotesize
\begin{verbatim}
make install-autostart-fd
\end{verbatim}
\normalsize

\subsection*{Other Make Notes}
\index[general]{Notes!Other Make }
\index[general]{Other Make Notes }
\addcontentsline{toc}{subsection}{Other Make Notes}

To simply build a new executable in any directory, enter: 

\footnotesize
\begin{verbatim}
make
\end{verbatim}
\normalsize

To clean out all the objects and binaries (including the files named 1, 2, or
3, which Kern uses as temporary files), enter: 

\footnotesize
\begin{verbatim}
make clean
\end{verbatim}
\normalsize

To really clean out everything for distribution, enter: 

\footnotesize
\begin{verbatim}
make distclean
\end{verbatim}
\normalsize

note, this cleans out the Makefiles and is normally done from the top level
directory to prepare for distribution of the source. To recover from this
state, you must redo the {\bf ./configure} in the top level directory, since
all the Makefiles will be deleted. 

To add a new file in a subdirectory, edit the Makefile.in in that directory,
then simply do a {\bf make}. In most cases, the make will rebuild the Makefile
from the new Makefile.in. In some case, you may need to issue the {\bf make} a
second time. In extreme cases, cd to the top level directory and enter: {\bf
make Makefiles}. 

To add dependencies: 

\footnotesize
\begin{verbatim}
make depend
\end{verbatim}
\normalsize

The {\bf make depend} appends the header file dependencies for each of the
object files to Makefile and Makefile.in. This command should be done in each
directory where you change the dependencies. Normally, it only needs to be run
when you add or delete source or header files. {\bf make depend} is normally
automatically invoked during the configuration process. 

To install: 

\footnotesize
\begin{verbatim}
make install
\end{verbatim}
\normalsize

This not normally done if you are developing Bacula, but is used if you are
going to run it to backup your system. 

After doing a {\bf make install} the following files will be installed on your
system (more or less). The exact files and location (directory) for each file
depends on your {\bf ./configure} command (e.g. gnome-console and
gnome-console.conf are not installed if you do not configure GNOME. Also, if
you are using SQLite instead of mysql, some of the files will be different). 

\footnotesize
\begin{verbatim}
bacula
bacula-dir
bacula-dir.conf
bacula-fd
bacula-fd.conf
bacula-sd
bacula-sd.conf
bacula-tray-monitor
tray-monitor.conf
bextract
bls
bscan
btape
btraceback
btraceback.gdb
bconsole
bconsole.conf
create_mysql_database
dbcheck
delete_catalog_backup
drop_bacula_tables
drop_mysql_tables
fd
gnome-console
gnome-console.conf
make_bacula_tables
make_catalog_backup
make_mysql_tables
mtx-changer
query.sql
bsmtp
startmysql
stopmysql
wx-console
wx-console.conf
\end{verbatim}
\normalsize

\label{monitor}

\subsection*{Installing Tray Monitor}
\index[general]{Monitor!Installing Tray }
\index[general]{Installing Tray Monitor }
\addcontentsline{toc}{subsection}{Installing Tray Monitor}

The Tray Monitor is already installed if you used the {\bf
\verb{--{enable-tray-monitor} configure option and ran {\bf make install}.

As you don't run your graphical environment as root (if you do, you should
change that bad habit), don't forget to allow your user to read {\bf
tray-monitor.conf}, and to execute {\bf bacula-tray-monitor} (this is not a
security issue).

Then log into your graphical environment (KDE, Gnome or something else), run
{\bf bacula-tray-monitor} as your user, and see if a cassette icon appear
somewhere on the screen, usually on the task bar.
If it doesn't, follow the instructions below related to your environment or
window manager. 

\subsubsection*{GNOME}
\index[general]{GNOME }
\addcontentsline{toc}{subsubsection}{GNOME}

System tray, or notification area if you use the GNOME terminology, has been
supported in GNOME since version 2.2. To activate it, right-click on one of
your panels, open the menu {\bf Add to this Panel}, then {\bf Utility} and
finally click on {\bf Notification Area}. 

\subsubsection*{KDE}
\index[general]{KDE }
\addcontentsline{toc}{subsubsection}{KDE}

System tray has been supported in KDE since version 3.1. To activate it,
right-click on one of your panels, open the menu {\bf Add}, then {\bf Applet}
and finally click on {\bf System Tray}. 

\subsubsection*{Other window managers}
\index[general]{Managers!Other window }
\index[general]{Other window managers }
\addcontentsline{toc}{subsubsection}{Other window managers}

Read the documentation to know if the Freedesktop system tray standard is
supported by your window manager, and if applicable, how to activate it. 

\subsection*{Modifying the Bacula Configuration Files}
\index[general]{Modifying the Bacula Configuration Files }
\index[general]{Files!Modifying the Bacula Configuration }
\addcontentsline{toc}{subsection}{Modifying the Bacula Configuration Files}

See the chapter 
\ilink{Configuring Bacula}{_ChapterStart16} in this manual for
instructions on how to set Bacula configuration files. 
