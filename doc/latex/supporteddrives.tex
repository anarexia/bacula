%%
%%

\section*{Supported Tape Drives}
\label{_ChapterStart19}
\index[general]{Drives!Supported Tape }
\index[general]{Supported Tape Drives }
\addcontentsline{toc}{section}{Supported Tape Drives}

\subsection*{Supported Tape Drives}
\label{SupportedDrives}
\index[general]{Drives!Supported Tape }
\index[general]{Supported Tape Drives }
\addcontentsline{toc}{subsection}{Supported Tape Drives}

Even if your drive is on the list below, please check the 
\ilink{Tape Testing Chapter}{btape1} of this manual for
procedures that you can use to verify if your tape drive will work with
Bacula. If your drive is in fixed block mode, it may appear to work with
Bacula until you attempt to do a restore and Bacula wants to position the
tape. You can be sure only by following the procedures suggested above and
testing. 

It is very difficult to supply a list of supported tape drives, or drives that
are known to work with Bacula because of limited feedback (so if you use
Bacula on a different drive, please let us know). Based on user feedback, the
following drives are known to work with Bacula. A dash in a column means
unknown: 

\addcontentsline{lot}{table}{Supported Tape Drives}
\begin{longtable}{|p{2.0in}|l|l|p{2.5in}|l|}
 \hline 
\multicolumn{1}{|c| }{\bf OS } & \multicolumn{1}{c| }{\bf Man. } &
\multicolumn{1}{c| }{\bf Media } & \multicolumn{1}{c| }{\bf Model } &
\multicolumn{1}{c| }{\bf Capacity  } \\
 \hline {- } & {ADIC } & {DLT } & {Adic Scalar 100 DLT } & {100GB  } \\
 \hline {- } & {ADIC } & {DLT } & {Adic Fastor 22 DLT } & {-  } \\
 \hline {- } & {- } & {DDS } & {Compaq DDS 2,3,4 } & {-  } \\
 \hline {- } & {Exabyte } & {-  } & {Exabyte drives less than 10 years old } & {-  } \\
 \hline {- } & {Exabyte } & {-  } & {Exabyte VXA drives } & {-  } \\
 \hline {- } & {HP } & {Travan 4 } & {Colorado T4000S } & {-  } \\
 \hline {- } & {HP } & {DLT } & {HP DLT drives } & {-  } \\
 \hline {- } & {HP } & {LTO } & {HP LTO Ultrium drives } & {-  } \\
 \hline {- } & {IBM} & {??} & {3480, 3480XL, 3490, 3490E, 3580 and 3590 drives} & {-  } \\
 \hline {FreeBSD 4.10 RELEASE } & {HP } & {DAT } & {HP StorageWorks DAT72i } & {-  } \\
 \hline {- } & {Overland } & {LTO } & {LoaderXpress LTO } & {-  } \\
 \hline {- } & {Overland } & {- } & {Neo2000 } & {-  } \\
 \hline {- } & {OnStream } & {- } & {OnStream drives (see below) } & {-  } \\
 \hline {- } & {Quantum } & {DLT } & {DLT-8000 } & {40/80GB  } \\
 \hline {Linux } & {Seagate } & {DDS-4 } & {Scorpio 40 } & {20/40GB  } \\
 \hline {FreeBSD 4.9 STABLE } & {Seagate } & {DDS-4 } & {STA2401LW } & {20/40GB  } \\
 \hline {FreeBSD 5.2.1 pthreads patched RELEASE } & {Seagate } & {AIT-1 } & {STA1701W} & {35/70GB  } \\
 \hline {Linux } & {Sony } & {DDS-2,3,4 } & {- } & {4-40GB  } \\
 \hline {Linux } & {Tandberg } & {- } & {Tandbert MLR3 } & {-  } \\
 \hline {FreeBSD } & {Tandberg } & {- } & {Tandberg SLR6 } & {-  } \\
 \hline {Solaris } & {Tandberg } & {- } & {Tandberg SLR75 } & {- } \\
 \hline 

\end{longtable}

There is a list of \ilink{supported autochangers}{Models} in the Supported
Autochangers chapter of this document, where you will find other tape drives
that work with Bacula. 

\subsection*{Unsupported Tape Drives}
\label{UnSupportedDrives}
\index[general]{Unsupported Tape Drives }
\index[general]{Drives!Unsupported Tape }
\addcontentsline{toc}{subsection}{Unsupported Tape Drives}

Previously OnStream IDE-SCSI tape drives did not work with Bacula. As of
Bacula version 1.33 and the osst kernel driver version 0.9.14 or later, they
now work. Please see the testing chapter as you must set a fixed block size. 

QIC tapes are known to have a number of particularities (fixed block size, and
one EOF rather than two to terminate the tape). As a consequence, you will
need to take a lot of care in configuring them to make them work correctly
with Bacula. 

\subsection*{FreeBSD Users Be Aware!!!}
\index[general]{FreeBSD Users Be Aware }
\index[general]{Aware!FreeBSD Users Be }
\addcontentsline{toc}{subsection}{FreeBSD Users Be Aware!!!}

Unless you have patched the pthreads library on most FreeBSD systems, you will
lose data when Bacula spans tapes. This is because the unpatched pthreads
library fails to return a warning status to Bacula that the end of the tape is
near. Please see the 
\ilink{Tape Testing Chapter}{FreeBSDTapes} of this manual for
{\bf important} information on how to configure your tape drive for
compatibility with Bacula. 

\subsection*{Supported Autochangers}
\index[general]{Autochangers!Supported }
\index[general]{Supported Autochangers }
\addcontentsline{toc}{subsection}{Supported Autochangers}

For information on supported autochangers, please see the 
\ilink{Autochangers Known to Work with Bacula}{Models}
section of the Supported Autochangers chapter of this manual. 
