%%
%%

\section*{Console Configuration}
\label{_ChapterStart36}
\index[general]{Configuration!Console }
\index[general]{Console Configuration }
\addcontentsline{toc}{section}{Console Configuration}

\subsection*{General}
\index[general]{General }
\addcontentsline{toc}{subsection}{General}

The Console configuration file is the simplest of all the configuration files,
and in general, you should not need to change it except for the password. It
simply contains the information necessary to contact the Director or
Directors. 

For a general discussion of configuration file and resources including the
data types recognized by {\bf Bacula}, please see the 
\ilink{Configuration}{_ChapterStart16} chapter of this manual. 

The following Console Resource definition must be defined: 

\begin{itemize}
\item 
   \ilink{Director}{DirectorResource3} -- to  define the
   Director's name and his access password. Note,  you may define more than one
Director resource in the  Console configuration file. If you do so, the
Console program  will ask you which one you want to use. 
\end{itemize}

\subsection*{The Director Resource}
\label{DirectorResource3}
\index[general]{Director Resource }
\index[general]{Resource!Director }
\addcontentsline{toc}{subsection}{Director Resource}

The Director resource defines the attributes of the Director running on the
network. You may have multiple Director resource specifications in a single
Console configuration file. If you have more than one, you will be prompted to
choose one when you start the {\bf Console} program. 

\begin{description}

\item [Director]
   \index[console]{Director }
   Start of the Director records. 

\item [Name = \lt{}name\gt{}]
   \index[console]{Name  }
   The director name used to select  among different Directors, otherwise, this
name is not used. 

\item [DIRPort = \lt{}port-number\gt{}]
   \index[dir]{DIRPort  }
   Specify the port to use to connect  to the Director. This value will most
likely already be set to the value  you specified on the {\bf
\verb{--{with-base-port} option of the  {\bf ./configure} command. This port must be
identical to the  {\bf DIRport} specified in the {\bf Director} resource of
the 
\ilink{Director's configuration}{_ChapterStart40} file.  The
default is 9101 so this record is not normally specified. 

\item [Address = \lt{}address\gt{}]
   \index[dir]{Address  }
   Where the address is a host name,  a fully qualified domain name, or a network
address used to connect  to the Director. 

\item [Password = \lt{}password\gt{}]
   \index[dir]{Password  }
   Where the password is the  password needed for the Director to accept the
Console connection.  This password must be identical to the {\bf Password}
specified in  the {\bf Director} resource of the 
\ilink{Director's configuration}{_ChapterStart40} file. This 
record is required. 
\end{description}

An actual example might be: 

\footnotesize
\begin{verbatim}
Director {
  Name = HeadMan
  address = rufus.cats.com
  password = xyz1erploit
}
\end{verbatim}
\normalsize

\subsection*{The ConsoleFont Resource}
\index[general]{Resource!ConsoleFont }
\index[general]{ConsoleFont Resource }
\addcontentsline{toc}{subsection}{ConsoleFont Resource}

The ConsoleFont resource is available only in the GNOME version of the
console. It permits you to define the font that you want used to display in
the main listing window. 

\begin{description}

\item [ConsoleFont]
   \index[console]{ConsoleFont }
   Start of the ConsoleFont records. 

\item [Name = \lt{}name\gt{}]
   \index[console]{Name  }
   The name of the font. 

\item [Font = \lt{}X-Window Font Specification\gt{}]
   \index[console]{Font  }
   The string value given here defines the desired font. It  is specified in the
standard cryptic X Window format. For  example, the default specification is: 

\footnotesize
\begin{verbatim}
Font = "-misc-fixed-medium-r-normal-*-*-130-*-*-c-*-iso8859-1"
\end{verbatim}
\normalsize

\end{description}

Thanks to Phil Stracchino for providing the code for this feature. 

An actual example might be: 

\footnotesize
\begin{verbatim}
ConsoleFont {
  Name = Default
Font = "-misc-fixed-medium-r-normal-*-*-130-*-*-c-*-iso8859-1"
}
\end{verbatim}
\normalsize

\subsection*{The Console Resource}
\label{ConsoleResource}
\index[general]{Console Resource }
\index[general]{Resource!Console }
\addcontentsline{toc}{subsection}{Console Resource}

As of Bacula version 1.33 and higher, there are three different kinds of
consoles, which the administrator or user can use to interact with the
Director. These three kinds of consoles comprise three different security
levels. 

\begin{itemize}
\item The first console type is an {\bf anonymous} or {\bf default}  console,
   which has full privileges. There is no console  resource necessary for this
   type since the password is  specified in the Director resource. This is the
kind of  console that was initially implemented in versions prior to  1.33 and
remains valid. Typically you would use it only  for administrators.  
\item The second type of console, and new to version 1.33 and  higher is a
   ``named'' console defined within a  Console resource in both the Director's
   configuration file  and in the Console's configuration file. Both the names 
and the passwords in these two entries must match much as  is the case for
Client programs. 

This second type of  console begins with absolutely no privileges except those
explicitly specified in the Director's Console resource.  Thus you can have
multiple Consoles with different names  and passwords, sort of like multiple
users, each with  different privileges. As a default, these consoles can do 
absolutely nothing -- no commands what so ever. You give  them privileges or
rather access to commands and resources  by specifying access control lists in
the Director's  Console resource. Note, if you are specifying such a  console,
you will want to put a null password in the  Director resource.  
\item The third type of console is similar to the above mentioned  one in that
   it requires a Console resource definition in  both the Director and the
   Console. In addition, if the  console name, provided on the {\bf Name =}
directive, is  the same as a Client name, the user of that console is 
permitted to use the {\bf SetIP} command to change the  Address directive in
the Director's client resource to the  IP address of the Console. This permits
portables or other  machines using DHCP (non-fixed IP addresses) to 
``notify'' the Director of their current IP  address. 
\end{itemize}

The Console resource is optional and need not be specified. However, if it is
specified, you can use ACLs (Access Control Lists) in the Director's
configuration file to restrict the particular console (or user) to see only
information pertaining to his jobs or client machine. 

The following configuration files were supplied by Phil Stracchino. For
example, if we define the following in the user's bconsole.conf file (or
perhaps the wx-console.conf file): 

\footnotesize
\begin{verbatim}
 Director {
   Name = MyDirector
   DIRport = 9101
   Address = myserver
   Password = "XXXXXXXXXXX"    # no, really.  this is not obfuscation.
 }
 
 Console {
   Name = restricted-user
   Password = "UntrustedUser"
 }
\end{verbatim}
\normalsize

Where the Password in the Director section is deliberately incorrect, and the
Console resource is given a name, in this case {\bf restricted-client}. Then
in the Director's bacula-dir.conf file (not directly accessible by the user),
we define: 

\footnotesize
\begin{verbatim}
Console {
  Name = restricted-user
  Password = "UntrustedUser"
  JobACL = "Restricted Client Save"
  ClientACL = restricted-client
  StorageACL = main-storage
  ScheduleACL = *all*
  PoolACL = *all*
  FileSetACL = "Restricted Client's FileSet"
  CatalogACL = DefaultCatalog
  CommandACL = run
}
\end{verbatim}
\normalsize

the user logging into the Director from his Console will get logged in as {\bf
restricted-client}, and he will only be able to see or access a Job with the
name {\bf Restricted Client Save} a Client with the name {\bf
restricted-client}, a Storage device {\bf main-storage}, any Schedule or Pool,
a FileSet named {\bf Restricted Client's File}, a Catalog named {\bf
DefaultCatalog}, and the only command he can use in the Console is the {\bf
run} command. In other words, this user is rather limited in what he can see
and do with Bacula. 

\subsection*{Console Commands}
\index[general]{Console Commands }
\index[general]{Commands!Console }
\addcontentsline{toc}{subsection}{Console Commands}

For more details on running the console and its commands, please see the 
\ilink{Bacula Console}{_ChapterStart23} chapter of this manual. 

\subsection*{Sample Console Configuration File}
\label{SampleConfiguration2}
\index[general]{File!Sample Console Configuration }
\index[general]{Sample Console Configuration File }
\addcontentsline{toc}{subsection}{Sample Console Configuration File}

A example Console configuration file might be the following: 

\footnotesize
\begin{verbatim}
#
# Bacula Console Configuration File
#
Director {
  Name = HeadMan
  address = "my_machine.my_domain.com"
  Password = Console_password
}
\end{verbatim}
\normalsize
