%%
%%

\section*{The Old FileSet Resource}
\label{_ChapterStart}
\label{FileSetResource}
\index[general]{Resource!Old FileSet }
\index[general]{Old FileSet Resource }
\addcontentsline{toc}{section}{Old FileSet Resource}

Note, this form of the FileSet resource still works but has been replaced by a
new more flexible form in Bacula version 1.34.3. As a consequence, you are
encouraged to convert to the new form as this one is deprecated and will be
removed in a future version. 

The FileSet resource defines what files are to be included in a backup job. At
least one {\bf FileSet} resource is required. It consists of a list of files
or directories to be included, a list of files or directories to be excluded
and the various backup options such as compression, encryption, and signatures
that are to be applied to each file. 

Any change to the list of the included files will cause Bacula to
automatically create a new FileSet (defined by the name and an MD5 checksum of
the Include contents). Each time a new FileSet is created, Bacula will ensure
that the first backup is always a Full save. 

\begin{description}

\item {\bf FileSet}
\index[dir]{FileSet }
Start of the FileSet records. At least one {\bf FileSet}  resource must be
defined. 

\item {\bf Name = \lt{}name\gt{}}
\index[dir]{Name  }
The name of the FileSet resource.  This record is required. 

\item {\bf Include = \lt{}processing-options\gt{} 
\ \ \{ \lt{}file-list\gt{} \} 
}
\index[dir]{Include  }

The Include resource specifies the list of files and/or directories to be
included in the backup job. There can be any number of {\bf Include} {\bf
file-list} specifications within the FileSet, each having its own set of {\bf
processing-options}. Normally, the {\bf file-list} consists of one file or
directory name per line. Directory names should be specified without a
trailing slash. Wild-card (or glob matching) does not work when used in an
Include list. It does work in an Exclude list though. Just the same, any
asterisk (*), question mark (?), or left-bracket ([) must be preceded by a
slash (\textbackslash{}\textbackslash{}) if you want it to represent the
literal character. 

You should {\bf always} specify a full path for every directory and file that
you list in the FileSet. In addition, on Windows machines, you should {\bf
always} prefix the directory or filename with the drive specification (e.g.
{\bf c:/xxx}) using Unix directory name separators (forward slash). However,
within an {\bf Exclude} where for some reason the exclude will not work with a
prefixed drive letter. If you want to specify a drive letter in exclusions on
Win32 systems, you can do so by specifying: 

\footnotesize
\begin{verbatim}
  Exclude = { /cygdrive/d/archive/Mulberry }
\end{verbatim}
\normalsize

where in this case, the {\bf /cygdrive/d} \&nsbp; is Cygwin's way of referring
to drives on Win32 (thanks to Mathieu Arnold for this tip). 

Bacula's default for processing directories is to recursively descend in the
directory saving all files and subdirectories. Bacula will not by default
cross file systems (or mount points in Unix parlance). This means that if you
specify the root partition (e.g. {\bf /}), Bacula will save only the root
partition and not any of the other mounted file systems. Similarly on Windows
systems, you must explicitly specify each of the drives you want saved (e.g.
{\bf c:/} and {\bf d:/} ...). In addition, at least for Windows systems, you
will most likely want to enclose each specification within double quotes. The
{\bf df} command on Unix systems will show you which mount points you must
specify to save everything. See below for an example. 

Take special care not to include a directory twice or Bacula will backup the
same files two times wasting a lot of space on your archive device. Including
a directory twice is very easy to do. For example: 

\footnotesize
\begin{verbatim}
  Include = { / /usr }
\end{verbatim}
\normalsize

on a Unix system where /usr is a subdirectory (rather than a mounted
filesystem) will cause /usr to be backed up twice. In this case, on Bacula
versions prior to 1.32f-5-09Mar04 due to a bug, you will not be able to
restore hard linked files that were backed up twice. 

The {\bf \lt{}processing-options\gt{}} is optional. If specified, it is a list
of {\bf keyword=value} options to be applied to the file-list. Multiple
options may be specified by separating them with spaces. These options are
used to modify the default processing behavior of the files included. Since
there can be multiple {\bf Include} sets, this permits effectively specifying
the desired options (compression, encryption, ...) on a file by file basis.
The options may be one of the following: 

\begin{description}

\item {\bf compression=GZIP}
\index[fd]{compression }
All files saved will be software  compressed using the GNU ZIP compression
format. The  compression is done on a file by file basis by the File daemon. 
If there is a problem reading the tape in a  single record of a file, it will
at most affect that file and none  of the other files on the tape. Normally
this option is {\bf not} needed  if you have a modern tape drive as the drive
will do its own  compression. However, compression is very important if you
are writing  your Volumes to a file, and it can also be helpful if you have a 
fast computer but a slow network.  

Specifying {\bf GZIP} uses the default compression level six  (i.e. {\bf GZIP}
is identical to {\bf GZIP6}). If you  want a different compression level (1
through 9), you can specify  it by appending the level number with no
intervening spaces  to {\bf GZIP}. Thus {\bf compression=GZIP1} would give
minimum  compression but the fastest algorithm, and {\bf compression=GZIP9} 
would give the highest level of compression, but requires more  computation.
According to the GZIP documentation, compression levels  greater than 6
generally give very little extra compression but are  rather CPU intensive. 

\item {\bf signature=MD5}
\index[fd]{signature }
An MD5 signature will be computed for all  files saved. Adding this option
generates about 5\% extra overhead  for each file saved. In addition to the
additional CPU time,  the MD5 signature adds 16 more bytes per file to your
catalog.  We strongly recommend that this option  be specified as a default
for all files. 

\item {\bf signature=SHA1}
\index[fd]{signature }
An SHA1 signature will be computed for all  The SHA1 algorithm is purported to
be some  what slower than the MD5 algorithm, but at the same time is 
significantly better from a cryptographic point of view (i.e.  much fewer
collisions, much lower probability of being hacked.)  It adds four more bytes
than the MD5 signature.  We strongly recommend that either this option  or MD5
be specified as a default for all files. Note, only  one of the two options
MD5 or SHA1 can be computed for any  file. 

\item {\bf *encryption=\lt{}algorithm\gt{}}
\index[fd]{*encryption }
All files saved will be  encrypted using one of the following algorithms (NOT
YET IMPLEMENTED):  

\begin{description}

\item {\bf *AES}
\index[fd]{*AES }
\end{description}

\item {\bf verify=\lt{}options\gt{}}
\index[fd]{verify }
The options letters specified are used  when running a {\bf Verify
Level=Catalog} job, and may be any  combination of the following:  

\begin{description}

\item {\bf i}
compare the inodes  

\item {\bf p}
compare the permission bits  

\item {\bf n}
compare the number of links  

\item {\bf u}
compare the user id  

\item {\bf g}
compare the group id  

\item {\bf s}
compare the size  

\item {\bf a}
compare the access time  

\item {\bf m}
compare the modification time (st\_mtime)  

\item {\bf c}
compare the change time (st\_ctime)  

\item {\bf s}
report file size decreases  

\item {\bf 5}
compare the MD5 signature  

\item {\bf 1}
compare the SHA1 signature  
\end{description}

A useful set of general options on the {\bf Level=Catalog}  verify is {\bf
pins5} i.e. compare permission bits, inodes, number  of links, size, and MD5
changes. 

\item {\bf onefs=yes|no}
\index[fd]{onefs }
If set to {\bf yes} (the default), {\bf Bacula}  will remain on a single file
system. That is it will not backup  file systems that are mounted on a
subdirectory.  In this case, you must explicitly list each file system you
want saved.  If you set this option to {\bf no}, Bacula will backup  all
mounted file systems (i.e. traverse mount points) that  are found within the
{\bf FileSet}. Thus if  you have NFS or Samba file systems mounted on a
directory included  in your FileSet, they will also be backed up. Normally, it
is  preferable to set {\bf onefs=yes} and to explicitly name  each file system
you want backed up.  See the example below for more details. 
\label{portable}

\item {\bf portable=yes|no}
\index[fd]{portable }
If set to {\bf yes} (default is  {\bf no}), the Bacula File daemon will backup
Win32 files  in a portable format. By default, this option is set to  {\bf
no}, which means that on Win32 systems, the data will  be backed up using
Windows API calls and on WinNT/2K/XP,  the security and ownership data will be
properly backed up  (and restored), but the data format is not portable to
other  systems -- e.g. Unix, Win95/98/Me. On Unix systems, this  option is
ignored, and unless you have a specific need to  have portable backups, we
recommend accept the default  ({\bf no}) so that the maximum information
concerning  your files is backed up. 

\item {\bf recurse=yes|no}
\index[fd]{recurse }
If set to {\bf yes} (the default),  Bacula will recurse (or descend) into all
subdirectories  found unless the directory is explicitly excluded  using an
{\bf exclude} definition.  If you set  {\bf recurse=no}, Bacula will save the
subdirectory entries,  but not descend into the subdirectories, and thus  will
not save the contents of the subdirectories. Normally, you  will want the
default ({\bf yes}).  

\item {\bf sparse=yes|no}
\index[dir]{sparse }
Enable special code that checks for sparse files  such as created by ndbm. The
default is {\bf no}, so no checks  are made for sparse files. You may specify
{\bf sparse=yes} even  on files that are not sparse file. No harm will be
done, but there  will be a small additional overhead to check for buffers of 
all zero, and a small additional amount of space on the output  archive will
be used to save the seek address of each non-zero  record read.  

{\bf Restrictions:} Bacula reads files in 32K buffers.  If the whole buffer is
zero, it will be treated as a sparse  block and not written to tape. However,
if any part of the buffer  is non-zero, the whole buffer will be written to
tape, possibly  including some disk sectors (generally 4098 bytes) that are
all  zero. As a consequence, Bacula's detection of sparse blocks is in  32K
increments rather than the system block size. If anyone  considers this to be
a real problem, please send in a request  for change with the reason. The
sparse code was first  implemented in version 1.27.  

If you are not familiar with sparse files, an example is  say a file where you
wrote 512 bytes at address zero, then  512 bytes at address 1 million. The
operating system will  allocate only two blocks, and the empty space or hole 
will have nothing allocated. However, when you read the  sparse file and read
the addresses where nothing was written,  the OS will return all zeros as if
the space were allocated,  and if you backup such a file, a lot of space will
be used  to write zeros to the volume. Worse yet, when you restore  the file,
all the previously empty space will now be allocated  using much more disk
space. By turning on the {\bf sparse}  option, Bacula will specifically look
for empty space in  the file, and any empty space will not be written to the
Volume,  nor will it be restored. The price to pay for this is that  Bacula
must search each block it reads before writing it.  On a slow system, this may
be important. If you suspect  you have sparse files, you should benchmark the
difference  or set sparse for only those files that are really sparse. 
\label{readfifo}

\item {\bf readfifo=yes|no}
\index[fd]{readfifo }
If enabled, tells the Client to  read the data on a backup and write the data
on a restore  to any FIFO (pipe) that is explicitly mentioned  in the FileSet.
In this case, you must have a program already  running that writes into the
FIFO for a backup or reads  from the FIFO on a restore. This can be
accomplished with  the {\bf RunBeforeJob} record. If this is not the case, 
Bacula will hang indefinitely on reading/writing the FIFO.  When this is not
enabled (default), the Client simply  saves the directory entry for the FIFO. 

\item {\bf mtimeonly=yes|no}
\index[dir]{mtimeonly }
If enabled, tells the Client that  the selection of files during Incremental
and Differential  backups should based only on the st\_mtime value in the
stat()  packet. The default is {\bf no} which means that the  selection of
files to be backed up will be based on both the  st\_mtime and the st\_ctime
values. In general, it is not  recommended to use this option. 

\item {\bf keepatime=yes|no}
\index[dir]{keepatime }
The default is {\bf no}. When  enabled, Bacula will reset the st\_atime
(access time) field  of files that it backs up to their value prior to the 
backup. This option is not generally recommended as there  are very few
programs that use st\_atime, and the backup  overhead is increased because of
the additional system  call necessary to reset the times. (I'm not sure this 
works on Win32). 
\end{description}

{\bf \lt{}file-list\gt{}} is a space separated list of filenames and/or
directory names. To include names containing spaces, enclose the name between
double-quotes. The list may span multiple lines, in fact, normally it is good
practice to specify each filename on a separate line. 

There are a number of special cases when specifying files or directories in a
{\bf file-list}. They are: 

\begin{itemize}
\item Any file-list item preceded by an at-sign (@) is assumed to be a 
filename containing a list of files, which is read when  the configuration
file is parsed during Director startup.  Note, that the file is read on the
Director's machine  and not on the Client.  
\item Any file-list item beginning with a vertical bar (|) is  assumed to be a
program. This program will be executed  on the Director's machine at the time
the Job starts (not  when the Director reads the configuration file), and any
output  from that program will be assumed to be a list of files or 
directories, one per line, to be included. This allows you to  have a job that
for example includes all the local partitions even  if you change the
partitioning by adding a disk. In general, you will  need to prefix your
command or commands with a {\bf sh -c} so that  they are invoked by a shell.
This will not be the case if you are  invoking a script as in the second
example below. Also, you must  take care to escape wild-cards and ensure that
any spaces in your  command are escaped as well. If you use a single quotes
(') within  a double quote (``), Bacula will treat everything between  the
single quotes as one field so it will not be necessary to  escape the spaces.
In general, getting all the quotes and escapes  correct is a real pain as you
can see by the next example. As a  consequence, it is often easier to put
everything in a file, and simply  us the file name within Bacula. In that case
the {\bf sh -c} will  not be necessary providing the first line of the file is
 {\bf \#!/bin/sh}.  

As an example: 

\footnotesize
\begin{verbatim}
 
Include = signature=SHA1 {
   "|sh -c 'df -l | grep \"^/dev/hd[ab]\" | grep -v \".*/tmp\" \
      | awk \"{print \\$6}\"'"
}
\end{verbatim}
\normalsize

will produce a list of all the local partitions on a RedHat Linux  system.
Note, the above line was split, but should normally  be written on one line. 
Quoting is a real problem because you must quote for Bacula  which consists of
preceding every \textbackslash{} and every '' with a \textbackslash{}, and 
you must also quote for the shell command. In the end, it is probably  easier
just to execute a small file with: 

\footnotesize
\begin{verbatim}
Include = signature=MD5 {
   "|my_partitions"
}
\end{verbatim}
\normalsize

where my\_partitions has: 

\footnotesize
\begin{verbatim}
#!/bin/sh
df -l | grep "^/dev/hd[ab]" | grep -v ".*/tmp" \
      | awk "{print \$6}"
\end{verbatim}
\normalsize

If the vertical bar (|) is preceded by a backslash as in \textbackslash{}|, 
the program will be executed on the Client's machine instead  of on the
Director's machine -- (this is implemented but  not tested, and very likely
will not work on Windows). 
\item Any file-list item preceded by a less-than sign (\lt{}) will be taken 
to be a file. This file will be read on the Director's machine  at the time
the Job starts, and the data will be assumed to be a  list of directories or
files, one per line, to be included. This  feature allows you to modify the
external file and change what  will be saved without stopping and restarting
Bacula as would be  necessary if using the @ modifier noted above.  

If you precede the less-than sign (\lt{}) with a backslash  as in
\textbackslash{}\lt{}, the file-list will be read on the Client machine 
instead of on the Director's machine (implemented but not  tested).  
\item If you explicitly specify a block device such as {\bf /dev/hda1},  then
Bacula (starting with version 1.28) will assume that this  is a raw partition
to be backed up. In this case, you are strongly  urged to specify a {\bf
sparse=yes} include option, otherwise, you  will save the whole partition
rather than just the actual data that  the partition contains. For example: 

\footnotesize
\begin{verbatim}
Include = signature=MD5 sparse=yes {
   /dev/hd6
}
\end{verbatim}
\normalsize

will backup the data in device /dev/hd6.  

Ludovic Strappazon has pointed out that this feature can be  used to backup a
full Microsoft Windows disk. Simply boot into  the system using a Linux Rescue
disk, then load a statically  linked Bacula as described in the 
\ilink{ Disaster Recovery Using Bacula}{rescue.tex#_ChapterStart} chapter of
this manual. Then  simply save the whole disk partition. In the case of a
disaster, you  can then restore the desired partition. 
\item If you explicitly specify a FIFO device name (created with mkfifo),  and
you add the option {\bf readfifo=yes} as an option, Bacula  will read the FIFO
and back its data up to the Volume. For  example: 

\footnotesize
\begin{verbatim}
Include = signature=SHA1 readfifo=yes {
   /home/abc/fifo
}
\end{verbatim}
\normalsize

if {\bf /home/abc/fifo} is a fifo device, Bacula will open the  fifo, read it,
and store all data thus obtained on the Volume.  Please note, you must have a
process on the system that is  writing into the fifo, or Bacula will hang, and
after one  minute of waiting, it will go on to the next file. The data  read
can be anything since Bacula treats it as a stream.  

This feature can be an excellent way to do a ``hot''  backup of a very large
database. You can use the {\bf RunBeforeJob}  to create the fifo and to start
a program that dynamically reads  your database and writes it to the fifo.
Bacula will then write  it to the Volume.  

During the restore operation, the inverse is  true, after Bacula creates the
fifo if there was any data stored  with it (no need to explicitly list it or
add any options), that  data will be written back to the fifo. As a
consequence, if  any such FIFOs exist in the fileset to be restored, you must 
ensure that there is a reader program or Bacula will block,  and after one
minute, Bacula will time out the write to the  fifo and move on to the next
file. 
\end{itemize}

The Exclude Files specifies the list of files and/or directories to be
excluded from the backup job. The {\bf \lt{}file-list\gt{}} is a comma or
space separated list of filenames and/or directory names. To exclude names
containing spaces, enclose the name between double-quotes. Most often each
filename is on a separate line. 

For exclusions on Windows systems, do not include a leading drive letter such
as {\bf c:}. This does not work. Any filename preceded by an at-sign (@) is
assumed to be a filename on the Director's machine containing a list of files.

\end{description}

The following is an example of a valid FileSet resource definition: 

\footnotesize
\begin{verbatim}
FileSet {
  Name = "Full Set"
  Include = compression=GZIP signature=SHA1 sparse=yes {
     @/etc/backup.list
  }
  Include = {
     /root/myfile
     /usr/lib/another_file
  }
  Exclude = { *.o }
}
\end{verbatim}
\normalsize

Note, in the above example, all the files contained in /etc/backup.list will
be compressed with GZIP compression, an SHA1 signature will be computed on the
file's contents (its data), and sparse file handling will apply. 

The two files /root/myfile and /usr/lib/another\_file will also be saved but
without any options. In addition, all files with the extension {\bf .o} will
be excluded from the file set (i.e. from the backup). 

Suppose you want to save everything except {\bf /tmp} on your system. Doing a
{\bf df} command, you get the following output: 

\footnotesize
\begin{verbatim}
[kern@rufus k]$ df
Filesystem      1k-blocks      Used Available Use% Mounted on
/dev/hda5         5044156    439232   4348692  10% /
/dev/hda1           62193      4935     54047   9% /boot
/dev/hda9        20161172   5524660  13612372  29% /home
/dev/hda2           62217      6843     52161  12% /rescue
/dev/hda8         5044156     42548   4745376   1% /tmp
/dev/hda6         5044156   2613132   2174792  55% /usr
none               127708         0    127708   0% /dev/shm
//minimatou/c$   14099200   9895424   4203776  71% /mnt/mmatou
lmatou:/          1554264    215884   1258056  15% /mnt/matou
lmatou:/home      2478140   1589952    760072  68% /mnt/matou/home
lmatou:/usr       1981000   1199960    678628  64% /mnt/matou/usr
lpmatou:/          995116    484112    459596  52% /mnt/pmatou
lpmatou:/home    19222656   2787880  15458228  16% /mnt/pmatou/home
lpmatou:/usr      2478140   2038764    311260  87% /mnt/pmatou/usr
deuter:/          4806936     97684   4465064   3% /mnt/deuter
deuter:/home      4806904    280100   4282620   7% /mnt/deuter/home
deuter:/files    44133352  27652876  14238608  67% /mnt/deuter/files
\end{verbatim}
\normalsize

Now, if you specify only {\bf /} in your Include list, Bacula will only save
the Filesystem {\bf /dev/hda5}. To save all file systems except {\bf /tmp}
with out including any of the Samba or NFS mounted systems, and explicitly
excluding a /tmp, /proc, .journal, and .autofsck, which you will not want to
be saved and restored, you can use the following: 

\footnotesize
\begin{verbatim}
FileSet {
  Name = Everything
  Include = {
     /
     /boot
     /home
     /rescue
     /usr
  }
  Exclude = {
     /proc
     /tmp
     .journal
     .autofsck
  }
}
\end{verbatim}
\normalsize

Since /tmp is on its own filesystem and it was not explicitly named in the
Include list, it is not really needed in the exclude list. It is better to
list it in the Exclude list for clarity, and in case the disks are changed so
that it is no longer in its own partition. 

Please be aware that allowing Bacula to traverse or change file systems can be
{\bf very} dangerous. For example, with the following: 

\footnotesize
\begin{verbatim}
FileSet {
  Name = "Bad example"
  Include = onefs=no {
     /mnt/matou
  }
}
\end{verbatim}
\normalsize

you will be backing up an NFS mounted partition ({\bf /mnt/matou}), and since
{\bf onefs} is set to {\bf no}, Bacula will traverse file systems. However, if
{\bf /mnt/matou} has the current machine's file systems mounted, as is often
the case, you will get yourself into a recursive loop and the backup will
never end. 

The following FileSet definition will backup a raw partition: 

\footnotesize
\begin{verbatim}
FileSet {
  Name = "RawPartition"
  Include = sparse=yes {
     /dev/hda2
  }
}
\end{verbatim}
\normalsize

Note, in backing up and restoring a raw partition, you should ensure that no
other process including the system is writing to that partition. As a
precaution, you are strongly urged to ensure that the raw partition is not
mounted or is mounted read-only. If necessary, this can be done using the {\bf
RunBeforeJob} record. 

\subsection*{Additional Considerations for Using Excludes on Windows}
\index[general]{Additional Considerations for Using Excludes on Windows }
\index[general]{Windows!Additional Considerations for Using Excludes on }
\addcontentsline{toc}{subsection}{Additional Considerations for Using Excludes
on Windows}

For exclude lists to work correctly on Windows, you must observe the following
rules: 

\begin{itemize}
\item Filenames are case sensitive, so you must use the correct case.  
\item To exclude a directory, you must not have a trailing slash on the 
directory name.  
\item If you have spaces in your filename, you must enclose the entire name 
in double-quote characters (``). Trying to use a backslash before  the space
will not work.  
\item You must not precede the excluded file or directory with a drive  letter
(such as {\bf c:}) otherwise it will not work. 
\end{itemize}

Thanks to Thiago Lima for summarizing the above items for us. If you are
having difficulties getting includes or excludes to work, you might want to
try using the {\bf estimate job=xxx listing} command documented in the 
\ilink{Console chapter}{console.tex#estimate} of this manual. 
\label{win32}

\subsection*{Windows Considerations for FileSets}
\index[general]{FileSets!Windows Considerations for }
\index[general]{Windows Considerations for FileSets }
\addcontentsline{toc}{subsection}{Windows Considerations for FileSets}

If you are entering Windows file names, the directory path may be preceded by
the drive and a colon (as in c:). However, the path separators must be
specified in Unix convention (i.e. forward slash (/)). If you wish to include
a quote in a file name, precede the quote with a backslash
(\textbackslash{}\textbackslash{}). For example you might use the following
for a Windows machine to backup the ''My Documents`` directory: 

\footnotesize
\begin{verbatim}
FileSet {
  Name = "Windows Set"
  Include = {
     "c:/My Documents"
  }
  Exclude = { *.obj *.exe }
}
\end{verbatim}
\normalsize

When using exclusion on Windows, do not use a drive prefix (i.e. {\bf c:}) as
it will prevent the exclusion from working. However, if you need to specify a
drive letter in exclusions on Win32 systems, you can do so by specifying: 

\footnotesize
\begin{verbatim}
  Exclude = { /cygdrive/d/archive/Mulberry }
\end{verbatim}
\normalsize

where in this case, the {\bf /cygdrive/d} is Cygwin's way of referring to
drive {\bf d:} (thanks to Mathieu Arnold for this tip). 

\subsection*{A Windows Example FileSet}
\index[general]{FileSet!Windows Example }
\index[general]{Windows Example FileSet }
\addcontentsline{toc}{subsection}{Windows Example FileSet}

The following example was contributed by Phil Stracchino: 

\footnotesize
\begin{verbatim}
This is my Windows 2000 fileset:
FileSet {
  Name = "Windows 2000 Full Set"
  Include = signature=MD5 {
    c:/
  }
# Most of these files are excluded not because we don't want
#  them, but because Win2K won't allow them to be backed up
#  except via proprietary Win32 API calls.
  Exclude = {
    "/Documents and Settings/*/Application Data/*/Profiles/*/*/
         Cache/*"
    "/Documents and Settings/*/Local Settings/Application Data/
         Microsoft/Windows/[Uu][Ss][Rr][Cc][Ll][Aa][Ss][Ss].*"
    "/Documents and Settings/*/[Nn][Tt][Uu][Ss][Ee][Rr].*"
    "/Documents and Settings/*/Cookies/*"
    "/Documents and Settings/*/Local Settings/History/*"
    "/Documents and Settings/*/Local Settings/
         Temporary Internet Files/*"
    "/Documents and Settings/*/Local Settings/Temp/*"
    "/WINNT/CSC"
    "/WINNT/security/logs/scepol.log"
    "/WINNT/system32/config/*"
    "/WINNT/msdownld.tmp/*"
    "/WINNT/Internet Logs/*"
    "/WINNT/$Nt*Uninstall*"
    "/WINNT/Temp/*"
    "/temp/*"
    "/tmp/*"
    "/pagefile.sys"
  }
}
\end{verbatim}
\normalsize

Note, the three line of the above Exclude were split to fit on the document
page, they should be written on a single line in real use. 
